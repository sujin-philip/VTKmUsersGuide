% -*- latex -*-

\chapter{Advanced Worklet Customization}
\label{chap:AdvancedWorklets}

\fix{This chapter should be split up.}

Chapter~\ref{chap:Worklets} describes the basics of creating and using
worklets. Many visualization algorithms can be implemented using VTK-m's
existing worklet types and features. However, new algorithms and designs
may require features not provided by VTK-m's current worklet set. In such
cases it is possible to directly design filters using the lower level
device adapter operations \fix{as described in section bla}. But by adding
features to the worklet mechanisms, new designs can be integrated better
with the other VTK-m features and can be repurposed in interesting ways for
other algorithms.

This chapter provides the information necessary to create new mechanisms
for worklets. It first describes the interface for getting data from the
control environment objects to the data passed to a worklet invocation and
back. It then describes how to modify these mechanisms to create new data
movement structures and new worklet types.


\section{Invocation Objects}
\label{sec:InvocationObjects}


\section{Creating New Worklet Types}
\label{sec:NewWorkletTypes}

\subsection{New Worklet Superclasses}
\label{sec:NewWorkletSuperclasses}

\subsection{Dispatch Workflow}
\label{sec:DispatchWorkflow}

\subsection{New Dispatch Classes}
\label{sec:NewDispatchClasses}
