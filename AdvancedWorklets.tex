% -*- latex -*-

\chapter{Advanced Worklet Customization}
\label{chap:AdvancedWorklets}

\fix{This chapter should be split up.}

Chapter~\ref{chap:Worklets} describes the basics of creating and using
worklets. Many visualization algorithms can be implemented using VTK-m's
existing worklet types and features. However, new algorithms and designs
may require features not provided by VTK-m's current worklet set. In such
cases it is possible to directly design filters using the lower level
device adapter operations \fix{as described in section bla}. But by adding
features to the worklet mechanisms, new designs can be integrated better
with the other VTK-m features and can be repurposed in interesting ways for
other algorithms.

This chapter provides the information necessary to create new mechanisms
for worklets. It first describes the interface for getting data from the
control environment objects to the data passed to a worklet invocation and
back. It then describes how to modify these mechanisms to create new data
movement structures and new worklet types.

\section{Transferring Arguments from Control to Execution}
\label{sec:TransferringArguments}

From the \controlsignature and \executionsignature defined in worklets,
VTK-m uses template meta-programming to build the code required to manage
data from control to execution environment. This management is handled by
three classes that provide type checking, transportation, and fetching.

\fix{I've been thinking that one more feature that these classes should
  provide is the ability to return the size of the domain. That would make
  things simpler and safer for getting the input domain size and checking
  the remaining domain sizes.}

\subsection{Type Checks}
\label{sec:TypeChecks}

\index{type~check|(}

Before attempting to move data from the control to the execution
environment, the VTK-m dispatchers check the input types to ensure that
they are compatible with the associated \controlsignature concept. This is
done with the \vtkmcontarg{TypeCheck} \textcode{struct}.

The \textidentifier{TypeCheck} \textcode{struct} is templated with two
parameters. The first parameter is a tag that identifies which check to
perform. The second parameter is the type of the control argument (after any
dynamic casts). The \textidentifier{TypeCheck} class contains a static
constant Boolean named \textcode{value} that is \textcode{true} if the type
in the second parameter is compatible with the tag in the first or
\textcode{false} otherwise.

Type checks are implemented with a defined type check tag (which, by
convention, is defined in the \vtkmcontarg{} namespace and starts with
\textcode{TypeCheckTag}) and a partial specialization of the
\vtkmcontarg{TypeCheck} structure. The following type checks (identified by
their tags) are provided in VTK-m.

\begin{description}
\item[\vtkmcontarg{TypeCheckTagArray}] \index{type~check!array} True if the
  type is a \vtkmcont{ArrayHandle}. \textidentifier{TypeCheckTagArray} also
  has a template parameter that is a type list. The
  \textidentifier{ArrayHandle} must also have a value type contained in
  this type list.
\item[\vtkmcontarg{TypeCheckTagExecObject}]
  \index{type~check!execution~object} True if the type is an execution
  object. All execution objects must derive from
  \vtkmexec{ExecutionObjectBase} and must be copyable through
  \textcode{memcpy} or similar mechanism.
\end{description}

Here are some trivial examples of using
\textidentifier{TypeCheck}. Typically these checks are done internally in
the base VTK-m dispatcher code, so these examples are for demonstration
only.

\vtkmlisting{Behavior of \protect\vtkmcontarg{TypeCheck}.}{TypeCheck.cxx}

\index{type~check|)}

\subsection{Transport}
\label{sec:Transport}

\index{transport|(}

After all the argument types are checked, the base dispatcher must load the
data into the execution environment before scheduling a job to run
there. This is done with the \vtkmcontarg{Transport} \textcode{struct}.

The \textidentifier{Transport} \textcode{struct} is templated with three
parameters. The first parameter is a tag that identifies which transport to
perform. The second parameter is the type of the control parameter (after any
dynamic casts). The third parameter is a device adapter tag for the device
on which the data will be loaded.

A \textidentifier{Transport} contains a \textcode{typedef} named
\textcode{ExecObjectType} that is the type used after data is moved to the
execution environment. A \textidentifier{Transport} also has a
\textcode{const} parenthesis operator that takes the control-side object
and the size of the domain and returns an execution-side object. This
operator is called in the control environment, and the returned object must
be ready to be passed to the execution environment.

Transports are implemented with a defined transport tag (which, by
convention, is defined in the \vtkmcontarg{} namespace and starts with
\textcode{TransportTag}) and a partial specialization of the
\vtkmcontarg{Transport} structure. The following transports (identified by
their tags) are provided in VTK-m.

\begin{description}
\item[\vtkmcontarg{TransportTagArrayIn}] \index{transport!input~array}
  Loads data from a \vtkmcont{ArrayHandle} onto the specified device using
  the array handle's \textcode{PrepareForInput} method. The returned
  execution object is an array portal.
\item[\vtkmcontarg{TransportTagArrayOut}] \index{transport!output~array}
  Allocates data onto the specified device for a \vtkmcont{ArrayHandle}
  using the array handle's \textcode{PrepareForOutput} method. The returned
  execution object is an array portal.
\item[\vtkmcontarg{TransportTagExecObject}]
  \index{transport!execution~object} Simply returns the given execution
  object, which should be ready to load onto the device.
\end{description}

Here are some trivial examples of using
\textidentifier{Transport}. Typically this movement is done internally in
the base VTK-m dispatcher code, so these examples are for demonstration
only.

\vtkmlisting{Behavior of \protect\vtkmcontarg{Transport}.}{Transport.cxx}

\index{transport|)}

\subsection{Fetch}
\label{sec:Fetch}

\index{fetch|(}

Before the function of a worklet is invoked, the VTK-m internals pull the
appropriate data out of the execution object and pass it to the worklet
function. A class named \vtkmexecarg{Fetch} is responsible for pulling this
data out and putting computed data in to the execution objects.

The \textidentifier{Fetch} \textcode{struct} is templated with four
parameters. The first parameter is a tag that identifies which type of
fetch to perform. The second parameter is a different tag that identifies
the aspect of the data to fetch. The third parameter is an
\textidentifier{Invocation} type that provides details about how the
worklet is being dispatched including a list of execution object parameters
passed to the invocation. The fourth parameter is a \vtkm{IdComponent} that
points to the invocation parameter that the data should be fetched from.

A \textidentifier{Fetch} contains a \textcode{typedef} named
\textcode{ValueType} that is the type of data that is passed to and from
the worklet function. A \textidentifier{Fetch} also has a pair of methods
named \textcode{Load} and \textcode{Store} that get data from and add data
to the execution object at a given domain or thread index.

\index{aspect|(}
\index{fetch!aspect|see{aspect}}

Fetches are specified with a pair of fetch and aspect tags. Fetch tags are by
convention defined in the \vtkmexecarg{} namespace and start with
\textcode{FetchTag}. Likewise, aspect tags are also defined in the
\vtkmexecarg{} namespace and start with \textcode{AspectTag}. The
\textidentifier{Fetch} \textcode{typedef} is partially specialized on these
two tags.

\index{aspect!default} The most common aspect tag is
\vtkmexecarg{AspectTagDefault}, and all fetch tags should have a
specialization of \vtkmexecarg{Fetch} with this tag. The following list of
fetch tags describes the execution objects they work with and the data they
pull for each aspect tag they support.

\fix{Don't forget to add index entries for both fetch and aspect where
  appropriate.}

\begin{description}
\item[\vtkmexecarg{FetchTagArrayDirectIn}] \index{fetch!direct input array}
  Loads data from an array portal. This fetch only supports the
  \textidentifier{AspectTagDefault} aspect. The \textcode{Load} gets data
  directly from the domain (thread) index. The \textcode{Store} does
  nothing.
\item[\vtkmexecarg{FetchTagArrayDirectOut}] \index{fetch!direct output array}
  Stores data to an array portal. This fetch only supports the
  \textidentifier{AspectTagDefault} aspect. The \textcode{Store} sets data
  directly to the domain (thread) index. The \textcode{Load} does nothing.
\item[\vtkmexecarg{FetchTagExecObject}] \index{fetch!execution object}
  Simply returns an execution object. This fetch only supports the
  \textidentifier{AspectTagDefault} aspect. The \textcode{Load} returns the
  executive object in the associated parameter. The \textcode{Store} does
  nothing.
\end{description}

In addition to the aforementioned aspect tags that are explicitly paired
with fetch tags, VTK-m also provides some aspect tags that either modify
the behavior of a general fetch or simply ignore the type of fetch.

\begin{description}
\item[\vtkmexecarg{AspectTagWorkIndex}] \index{aspect!work index} Simply
  returns the domain (or thread) index ignoring any associated data. This
  aspect is used to implement the \sigtag{WorkIndex} execution signature
  tag.
\end{description}

\index{aspect|)}
\index{fetch|)}


\section{Function Interface Objects}
\label{sec:FunctionInterfaceObjects}

\index{function~interface|(}

For flexibility's sake a worklet is free to declare a \controlsignature
with whatever number of arguments are sensible for its operation. The
\textcode{Invoke} method of the dispatcher is expected to support arguments
that match these arguments, and part of the dispatching operation may
require these arguments to be augmented before the worklet is
scheduled. This leaves dispatchers with the tricky task of managing some
collection of arguments of unknown size and unknown types.

\fix{\textidentifier{FunctionInterface} is in the \vtkminternal{}
  interface. I still can't decide if it should be moved to the \vtkm{}
  interface.}

To simplify this management, VTK-m has the \vtkminternal{FunctionInterface}
class. \textidentifier{FunctionInterface} is a templated class that manages
a generic set of arguments and return value from a function. An instance of
\textidentifier{FunctionInterface} holds an instance of each argument. You
can apply the arguments in a \textidentifier{FunctionInterface} object to a
functor of a compatible prototype, and the resulting value of the function
call is saved in the \textidentifier{FunctionInterface}.

\subsection{Declaring and Creating}

\vtkminternal{FunctionInterface} is a templated class with a single
parameter. The parameter is the \index{function~signature}
\index{signature} \keyterm{signature} of the function. A signature is a
function type. The syntax in C++ is the return type followed by the
argument types encased in parentheses.

\vtkmlisting{Declaring \protect\vtkminternal{FunctionInterface}.}{DefineFunctionInterface.cxx}

The \vtkminternal{make\_FunctionInterface} function provies an easy way to
create a \textidentifier{FunctionInterface} and initialize the state of all
the parameters. \textidentifier{make\_FunctionInterface} takes a variable
number of arguments, one for each parameter. Since the return type is not
specified as an argument, you must always specify it as a template
parameter.

\vtkmlisting{Using \protect\vtkminternal{make\_FunctionInterface}.}{UseMakeFunctionInterface.cxx}

\subsection{Parameters}

One created, \textidentifier{FunctionInterface} contains methods to query
and manage the parameters and objects associated with them. The number of
parameters can be retrieved either with the constant field \index{arity}
\textcode{ARITY} or with the \textcode{GetArity} method.

\vtkmlisting{Getting the arity of a \textidentifier{FunctionInterface}.}{FunctionInterfaceArity.cxx}

To get a particular parameter, \textidentifier{FunctionInterface} has the
templated method \textcode{GetParameter}. The template parameter is the
index of the parameter. Note that the parameters in
\textidentifier{FunctionInterface} start at index 1. Although this is
uncommon in C++, it is customary to number function arguments starting at
1.

There are two ways to specify the index for \textcode{GetParameter}. The
first is to directly specify the template parameter (e.g.
\textcode{GetParameter<1>()}). However, note that in a templated function
or method where the type is not fully resolved the compiler will not
register \textcode{GetParameter} as a templated method and will fail to
parse the template argument without a \textcode{template} keyword. The
second way to specify the index is to provide a \vtkminternal{IndexTag}
object as an argument to \textcode{GetParameter}. Although this syntax is
more verbose, it works the same whether the
\textidentifier{FunctionInterface} is fully resolved or not. The following
example shows both methods in action.

\vtkmlisting{Using \textidentifier{FunctionInterface}\textcode{::GetParameter().}}{FunctionInterfaceGetParameter.cxx}

Likewise, there is a \textcode{SetParmeter} method for changing parameters.
The same rules for indexing and template specification apply.

\vtkmlisting{Using \textidentifier{FunctionInterface}\textcode{::SetParameter().}}{FunctionInterfaceSetParameter.cxx}

\subsection{Invoking}

\index{function~interface!invoke|(}

\textidentifier{FunctionInterface} can invoke a functor of a matching
signature using the parameters stored within. If the functor returns a
value, that return value will be stored in the
\textidentifier{FunctionInterface} object for later retrieval. There are
several versions of the invoke method. There are always seperate versions
of invoke methods for the control and execution environments so that
functors for either environment can be executed. The basic version of
invoke passes the parameters directly to the function and directly stores
the result.

\vtkmlisting{Invoking a \textidentifier{FunctionInterface}.}{FunctionInterfaceBasicInvoke.cxx}

Another form of the invoke methods takes a second transform functor that is
applied to each argument before passed to the main function. If the main
function returns a value, the transform is applied to that as well before
being stored back in the \textidentifier{FunctionInterface}.

\vtkmlisting{Invoking a \textidentifier{FunctionInterface} with a transform.}{FunctionInterfaceTransformInvoke.cxx}

\index{function~interface!invoke|)}

As demonstrated in the previous examples,
\textidentifier{FunctionInterface} has a method named
\textcode{GetReturnValue} that returns the value from the last invoke. Care
should be taken to only use \textcode{GetReturnValue} when the function
specification has a return value. If the function signature has a
\textcode{void} return type, using \textcode{GetReturnValue} will cause a
compile error.

\textidentifier{FunctionInterface} has an alternate method named
\textcode{GetReturnValueSafe} that returns the value wrapped in a templated
structure named \vtkminternal{FunctionInterfaceReturnContainer}. This
structure always has a static constant Boolean named \textcode{VALID} that
is \textcode{false} if there is no return type and \textcode{true}
otherwise. If the container is valid, it also has an entry named
\textcode{Value} containing the result.

\vtkmlisting{Getting return value from \textidentifier{FunctionInterface} safely.}{FunctionInterfaceReturnContainer.cxx}

\subsection{Modifying Parameters}

In addition to storing and querying parameters and invoking functions,
\textidentifier{FunctionInterface} also contains multiple ways to modify
the parameters to augment the function calls. This can be used in the same
use case as a chain of function calls that generally pass their parameters
but also augment the data along the way.

\index{function~interface!append parameter|(}

The \textcode{Append} method returns a new
\textidentifier{FunctionInterface} object with the same parameters plus a
new parameter (the argument to \textcode{Append}) to the end of the
parameters. There is also a matching \textcode{AppendType} templated
structure that can return the type of an augmented
\textidentifier{FunctionInterface} with a new type appended.

\vtkmlisting{Appending parameters to a \textidentifier{FunctionInterface}.}{FunctionInterfaceAppend.cxx}

\index{function~interface!append parameter|)}

\index{function~interface!replace parameter|(}

\textcode{Replace} is a similar method that returns a new
\textidentifier{FunctionInterface} object with the same paraemters except
with a specified parameter replaced with a new parameter (the argument to
\textcode{Replace}). There is also a matching \textcode{ReplaceType}
templated structure that can return the type of an augmented
\textidentifier{FunctionInterface} with one of the parameters replaced.

\vtkmlisting{Replacing parameters in a \textidentifier{FunctionInterface}.}{FunctionInterfaceReplace.cxx}

\index{function~interface!replace parameter|)}

It is sometimes desirable to make multiple modifications at a time. This
can be achieved by chaining modifications by calling \textcode{Append} or
\textcode{Replace} on the result of a previous call.

\vtkmlisting{Chaining \textcode{Replace} and \textcode{Append} with a \textidentifier{FunctionInterface}.}{FunctionInterfaceAppendAndReplace.cxx}

\subsection{Transformations}

Rather than replace a single item in a \textidentifier{FunctionInterface},
it is sometimes desirable to change them all in a similar
way. \textidentifier{FunctionInterface} supports two basic transform
operations on its parameters: a static transform and a dynamic
transform. The static transform determines its types at compile-time
whereas the dynamic transform happens at run-time.

\index{function~interface!static transform|(}

The static transform methods (named \textcode{StaticTransformCont} and
\textcode{StaticTransformExec}) operate by accepting a functor that defines
a function with two arguments. The first argument is the
\textidentifier{FunctionInterface} parameter to transform. The second
argument is an instance of the \vtkminternal{IndexTag} templated class that
statically identifies the parameter index being transformed. An
\textidentifier{IndexTag} object has no state, but the class contains a
static integer named \textidentifier{INDEX}. The function returns the
transformed argument.

The functor must also contain a templated class named \textcode{ReturnType}
with an internal type named \textcode{type} that defines the return type of
the transform for a given parameter type. \textcode{ReturnType} must have
two template parameters. The first template parameter is the type of the
\textidentifier{FunctionInterface} parameter to transform. It is the same
type as passed to the operator. The second template parameter is a
\vtkm{IdComponent} specifying the index.

The transformation is only applied to the parameters of the function. The
return argument is unaffected.

The return type can be determined with the \textcode{StaticTransformType}
template in the \textidentifier{FunctionInterface}
class. \textcode{StaticTransformType} has a single parameter that is the
transform functor and contains a type named \textcode{type} that is the
transformed \textidentifier{FunctionInterface}.

In the following example, a static transform is used to convert a
\textidentifier{FunctionInterface} to a new object that has the pointers to
the parameters rather than the values themselves. The parameter index is
always ignored as all parameters are uniformly transformed.

\vtkmlisting{Using a static transform of function interface class.}{FunctionInterfaceStaticTransform.cxx}

\index{function~interface!static transform|)}

\index{function~interface!dynamic transform|(}

There are cases where one set of parameters must be transformed to another
set, but the types of the new set are not known until run-time. That is,
the transformed type depends on the contents of the data. The
\textcode{DynamicTransformCont} method achieves this using a templated
callback that gets called with the correct type at run-time.

The dynamic transform works with two functors provided by the user code (as
opposed to the one functor in static transform). These functors are called
the transform functor and the finish functor. The transform functor accepts
three arguments. The first argument is a parameter to transform. The second
argument is a continue function. Rather than return the transformed value,
the transform functor calls the continue function, passing the transformed
value as an argument. The third argument is a \vtkminternal{IndexTag} for
the index of the argument being transformed.

Unlike its static counterpart, the dynamic transform method does not return
the transformed \textidentifier{FunctionInterface}. Instead, it passes the
transformed \textidentifier{FunctionInterface} to the finish functor passed
into \textcode{DynamicTransformCont}.

In the following contrived but illustrative example, a dynamic transform is
used to convert strings containing numbers into number arguments. Strings
that do not have numbers and all other arguments are passed through. Note
that because the types for strings are not determined till run-time, this
transform cannot be determined at compile time with meta-template
programming. The index argument is ignored because all arguments are
transformed the same way.


\vtkmlisting{Using a dynamic transform of a function interface.}{FunctionInterfaceDynamicTransform.cxx}

One common use for the \textidentifier{FunctionInterface} dynamic transform
is to convert parameters of virtual polymorphic type like
\vtkmcont{DynamicArrayHandle} and \vtkmcont{DynamicPointCoordinates}. This
use case is handled with a functor named
\vtkmcontinternal{DynamicTransform}. When used as the dynamic transform
functor, it will convert all of these dynamic types to their static
counterparts.

\vtkmlisting{Using \textidentifier{DynamicTransform} to cast dynamic arrays in a function interface.}{DynamicTransform.cxx}

\index{function~interface!dynamic transform|)}

\subsection{For Each}
\label{sec:FunctionInterface:ForEach}

\index{function~interface!for~each|(}

The invoke methods (principally) make a single function call passing all of
the parameters to this function. The transform methods call a function on
each parameter to convert it to some other data type. It is also sometimes
helpful to be able to call a unary function on each parameter that is not
expected to return a value. Typically the use case is for the function to
have some sort of side effect. For example, the function might print out
some value (such as in the following example) or perform some check on the
data and throw an exception on failure.

This feature is implemented in the for each methods of
\textidentifier{FunctionInterface}.  As with all the
\textidentifier{FunctionInterface} methods that take functors, there are
separate implementations for the control environment and the execution
environment. There are also separate implementations taking
\textcode{const} and non-\textcode{const} references to functors to
simplify making functors with side effects.

\vtkmlisting{Using the \textcode{ForEach} feature of \textidentifier{FunctionInterface}.}{FunctionInterfaceForEach.cxx}

\index{function~interface!for~each|)}

\index{function~interface|)}


\section{Invocation Objects}
\label{sec:InvocationObjects}


\section{Creating New \protect\controlsignature Tags}
\label{sec:NewControlSignatureTags}


\section{Creating New \protect\executionsignature Tags}
\label{sec:NewExecutionSignatureTags}


\section{Creating New Worklet Types}
\label{sec:NewWorkletTypes}

\subsection{New Worklet Superclasses}
\label{sec:NewWorkletSuperclasses}

\subsection{Dispatch Workflow}
\label{sec:DispatchWorkflow}

\subsection{New Dispatch Classes}
\label{sec:NewDispatchClasses}
