% -*- latex -*-

\chapter{Coding Conventions}
\label{chap:CodingConventions}

Several developers contribute to VTK-m and we welcome others who are
interested to also contribute to the project. To ensure readability and
consistency in the code, we have adopted the following coding
conventions. Many of these conventions are adapted from the coding
conventions of the VTK project. This is because many of the developers are
familiar with VTK coding and because we expect VTK-m to have continual
interaction with VTK.

\begin{itemize}
\item All code contributed to VTK-m must be compatible with VTK-m's BSD
  license.
\item Copyright notices should appear at the top of all source,
  configuration, and text files. The statement should have the following
  form:
\small\begin{verbatim}
//============================================================================
//  Copyright (c) Kitware, Inc.
//  All rights reserved.
//  See LICENSE.txt for details.
//  This software is distributed WITHOUT ANY WARRANTY; without even
//  the implied warranty of MERCHANTABILITY or FITNESS FOR A PARTICULAR
//  PURPOSE.  See the above copyright notice for more information.
//
//  Copyright 2014 Sandia Corporation.
//  Copyright 2014 UT-Battelle, LLC.
//  Copyright 2014. Los Alamos National Security
//
//  Under the terms of Contract DE-AC04-94AL85000 with Sandia Corporation,
//  the U.S. Government retains certain rights in this software.
//
//  Under the terms of Contract DE-AC52-06NA25396 with Los Alamos National
//  Laboratory (LANL), the U.S. Government retains certain rights in
//  this software.
//============================================================================
\end{verbatim}
  The \textfilename{CopyrightStatement} test checks all files for a similar
  statement. The test will print out a suggested text that can be copied
  and pasted to any file that has a missing copyright statement (with
  appropriate replacement of comment prefix). Exceptions to this copyright
  statement (for example, third-party files with different but compatible
  statements) can be added to \textfilename{LICENSE.txt}.
\item All include files should use include guards. starting right after the
  copyright statement. The naming convention of the include guard macro is
  that it should start with \textcode{vtk\_m} be followed with the path
  name, starting from the top-level source code directory under
  \textfilename{vtkm}, with non alphanumeric characters, such as
  \textcode{/} and \textcode{.} replaced with underscores. The
  \textcode{\#endif} part of the guard at the bottom of the file should
  include the guard name in a comment. For example, the
  \vtkmheader{vtkm/cont}{ArrayHandle.h} header contains the guard
\begin{verbatim}
#ifndef vtk_m_cont_ArrayHandle_h
#define vtk_m_cont_ArrayHandle_h
\end{verbatim}
  at the top and
\begin{verbatim}
#endif //vtk_m_cont_ArrayHandle_h
\end{verbatim}
\item VTK-m has several nested namespaces. The declaration of each
  namespace should be on its own line, and the code inside the namespace
  bracket should not be indented. The closing brace at the bottom of the
  namespace should be documented with a comment identifying the
  namespace. Namespaces can be grouped as desired. The following is a valid
  use of namespaces.
\begin{verbatim}
namespace vtkm {
namespace cont {

namespace detail {

class InternalClass;

} // namespace detail

class ExposedClass;

}
} // namespace vtkm::cont
\end{verbatim}
\item Multiple inheritance is not allowed in VTK-m classes.
\item Any functional public class should be in its own header file with the
  same name as the class. The file should be in a directory that
  corresponds to the namespace the class is in. There are several
  exceptions to this rule.
  \begin{itemize}
  \item Templated classes and template specialization often require the
    implementation of the class to be broken into pieces. Sometimes a
    specialization is placed in a header with a different name.
  \item Many VTK-m toolkit features are not encapsulated in
    classes. Functions may be collected by purpose or co-located with
    associated class.
  \item Although tags are technically classes, they behave as an
    enumeration for the compiler. Multiple tags that make up this
    enumeration are collected together.
  \item Some classes, such as \vtkm{Tuple} are meant to behave as basic
    types. These are sometimes collected together as if they were related
    \textcode{typedef}s. The \vtkmheader{vtkm}{Types.h} header is a good
    example of this.
  \end{itemize}
\item The indentation style can be characterized as the ``hanging brace''
  \fix{?}  (a modified whitesmith \fix{wrong, look it up})
  style. Indentations are two spaces, and the curly brace (scope delimiter)
  is placed on the following line and indented with the outer scope
  (i.e. the curly brace does not line up with the code in the block). This
  differs from VTK style, but was agreed on by the developers as the more
  common style.
\item Conditional clauses (including loop conditionals such as
  \textcode{for} and \textcode{while}) must be in braces below the
  conditional. That is, instead of
\begin{verbatim}
if (test) { clause; }
\end{verbatim}
  use
\begin{verbatim}
if (test)
  {
  clause;
  }
\end{verbatim}
  The rational for this requirement is to make it obvious whether the
  clause is executed when stepping through the code with the debugger. The
  one exception to this rule is when the clause contains a control-flow
  statement with obvious side effects such as \textcode{return} or
  \textcode{break}. However, even if the clause contains a single statement
  and is on the same line, the clause should be surrounded by braces.
\item Use two space indentation.
\item Tabs are not allowed. Only use spaces for indentation. No one can
  agree on what the size of a tab stop is, so it is better to not use them
  at all.
\item There should be no trailing whitespace in any line.
\item Use only alphanumeric characters in names. Use capitalization to
  demarcate words within a name (camel case). The exception is preprocessor
  macros and constant numbers that are, by convention, represented in all
  caps and a single underscore to demarcate words.
\item Namespace names are in all lowercase. They should be a single word
  that designates its meaning.
\item All class, method, member variable, and functions should start with a
  capital letter. Local variables should start in lower case and then use
  camel case. Exceptions can be made when such naming would conflict with
  previously established conventions in other library. (For example,
  \textcode{make\_ArrayHandle} corresponds to \textcode{make\_pair} in the
  standard template library.)
\item Always spell out words in names; do not use abbreviations except in
  cases where the shortened form is widely understood and a name in its own
  right (e.g. OpenMP).
\item Always use descriptive names in all identifiers, including local
  variable names. Particularly avoid meaningless names of a few characters
  (e.g. \textcode{x}, \textcode{foo}, or \textcode{tmp}) or numbered names
  with no meaning to the number or order (e.g. \textcode{value1},
  \textcode{value2},\ldots). Also avoid the meaningless for loop variable
  names \textcode{i}, \textcode{j}, \textcode{k}, etc. Instead, use a name
  that identifies what type of index is being referenced such as
  \textcode{pointIndex}, \textcode{vertexIndex}, \textcode{componentIndex},
  etc.
\item Classes are documented with Doxygen-style comments before classes,
  methods, and functions.
\item Exposed classes should not have public instance variables outside of
  exceptional situations. Access is given by convention through methods
  with names starting with \textcode{Set} and \textcode{Get} or through
  overloaded operators.
\item References to classes and functions should be fully qualified with
  the namespace. This makes it easier to establish classes and functions
  from different packages and to find source and documentation for the
  referenced class. As an exception, if one class references an internal or
  detail class clearly associated with it, the reference can be shortened
  to \textcode{internal::} or \textcode{detail::}.
\item use \textcode{this->} inside of methods when accessing class methods
  and instance variables to distinguish between local variables and
  instance variables.
\item Include statements should generally be in alphabetical order. They
  can be grouped by package and type.
\item Namespaces should not be brought into global scope or the scope of
  any VTK-m package namespace with the ``using'' keyword. It should also be
  avoided in class, method, and function scopes (fully qualified namespace
  references are preferred).
\item All code must be valid by the C++03 and C++11 specifications. It must
  also compile on older compilers that support C++98. Code that uses
  language features not available in C++98 must have a second
  implementation that works around the limitations of C++98. The
  \cmakevar{VTKM\_FORCE\_ANSI} turns on a compiler check for ANSI
  compatibility in gcc and clang compilers.
\item Limit all lines to 80 characters whenever possible.
\item New code must include regression tests that will run on the
  dashboards. Generally a new class will have an associated ``UnitTest''
  that will test the operation of the test directly. There may be other
  tests necessary that exercise the operation with different components or
  on different architectures.
\item All code must compile and run without error or warning messages on
  the nightly dashboards, which should include Windows, Mac, and Linux.
\item Use \vtkm{Scalar} in lieu of \textcode{float} or \textcode{double} to
  represent data for computation and use \vtkm{Id} in lieu of \textcode{int}
  or \textcode{long} for data structure indices. This future-proofs code
  against changes in precision of the architecture. The indices of
  \vtkm{Tuple} are an exception. Using \textcode{int} to reference
  \vtkm{Tuple} (and other related classes like \vtkmexec{CellField} and
  \vtkmmath{Matrix}) indices are acceptable as it is unreasonable to make
  these vectors longer than the precision of \textcode{int}.
\item All functions and methods defined within the Dax toolkit should be
  declared with \vtkmcontexport, \vtkmexecexport, or \vtkmexeccontexport.
\end{itemize}

We should note that although these conventions impose a strict statute on
VTK-m coding, these rules (other than those involving licensing and
copyright) are not meant to be dogmatic. Examples can be found in the
existing code that break these conventions, particularly when the
conventions stand in the way of readability (which is the point in having
them in the first place). For example, it is often the case that it is more
readable for a complicated \textcode{typedef} to stretch a few characters
past 80 even if it pushes past the end of a display.
