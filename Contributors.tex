\vspace*{\fill}
\begin{centering}
\includegraphics[height=0.6in]{images/SandiaLogo}\hfill
\includegraphics[height=0.6in]{images/DOELogo}\\
\includegraphics[width=8cm]{images/KitwareLogo}

Published by Kitware Inc. \copyright  2016\\

All product names mentioned herein are the trademarks of their respective owners. \\

This document is available under a Creative Commons Attribution 4.0
International license available at \fix{add url}% \url{http://www.paraview.org/download/}.

This project has been funded in whole or in part with Federal funds from
the Department of Energy, including from Sandia National Laboratories, Los
Alamos National Laboratory, Advanced Simulation and Computing, and Oak
Ridge National Laboratory.

Sandia National Laboratories is a multi-program laboratory managed and
operated by Sandia Corporation, a wholly owned subsidiary of Lockheed
Martin Corporation, for the U.S. Department of Energy's National Nuclear
Security Administration under contract DE-AC04-94AL85000.

Printed and produced in the United States of America.\\
ISBN number \fix{FILL IN ISBN NUMBERS HERE}\\
\end{centering}
\vspace*{\fill}

\newpage
\vspace*{\fill}
\hrulefill \\


\begin{centering}
{\bf CONTRIBUTORS}\\
\vspace{2em}
\end{centering}

This book includes contributions from the VTK-m community including the
VTK-m development team and the user community.
%% We would like to thank the following people for their significant
%% contributions to this updated text:

%% {\bf Cory Quammen}, {\bf Berk
%% Geveci}, {\bf Ben Boeckel}, {\bf Dave DeMarle}, and {\bf Shawn Waldon} from Kitware for their contributions throughout the text.

%% {\bf Ken Moreland}, for large sections of text from his excellent \emph{The ParaView Tutorial}\cite{ParaViewTutorial} that formed the basis of Chapter~\ref{chapter:ParallelDataVisualization}.
%% {\bf Thomas Maxwell} from \NASA,
%% {\bf John Patchett}, {\bf Boonthanome Nouanesengsy}, and {\bf James Ahrens} from \LANL, {\bf Bill
%% Sherman} from Indiana University, {\bf Nikhil Shetty} from the University of Wyoming, {\bf Aashish Choudhary} from %
%% \ifthenelse{\boolean{COMMUNITYEDITION}}{%
%% Kitware.
%% }%
%% {%
%% Kitware, and {\bf Eric Whiting} from Idaho National Laboratory, who contributed Chapter~\ref{chapter:CaseStudies}.
%%  {\bf George Zagaris} from Lawrence Livermore National Laboratory who was the primary author of Chapter~\ref{chapter:AMRvis}, and {\bf Andy Bauer} from Kitware who was the primary author for Chapter~\ref{chapter:CFDPostProcessing}.
%% }%
%% {\bf Burlen Loring} from Lawrence Berkeley National Laboratory authored wiki posts that were made into Chapter~\ref{chapter:MemoryInspector}.

%% Special thanks to the Kitware communications team including {\bf Katie
%% Osterdahl}, {\bf Sandy McKenzie}, {\bf Lisa Avila}, and {\bf Steve Jordan}.
%% This guide would not have been possible without their patient and persistent
%% efforts.\\


\begin{centering}
{\bf ABOUT THE COVER}\\
\vspace{2em}
\end{centering}

%% The cover image is a visualization of magnetic reconnection from the VPIC
%% project at Los Alamos National Laboratory. The visualization is generated from a
%% structured data containing 3.3 billion cells with two vector fields and one
%% scalar field, produced using 256 cores running ParaView on the interactive queue
%% on the Kraken at the National Institute for Computational Sciences (NICS).

%% The interior cover image is courtesy of Renato N. Elias, Associate Researcher at
%% the CFD Group from NACAD/COPPE/UFRJ, Rio de Janeiro, Brazil.

%% The cover design was done by Steve Jordan.

\hrulefill \\

\begin{centering}
Join the VTK-m Commuity at \url{m.vtk.org}\\
\end{centering}
\vspace*{\fill}
