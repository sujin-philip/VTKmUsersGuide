% -*- latex -*-

\chapter{Control Environment}
\label{chap:ControlEnvironment}

\index{control~environment|(}

The control environment is where code interfaces with applications and I/O
devices. The associated API is designed for users that want to use VTK-m to
analyze their data using provided or supplied worklets. Code for the
control environment is designed to run on a single thread (or one single
thread per process in an MPI job).

Most users of VTK-m will have some interaction with the control
environment, for you cannot define data structures or execute any
algorithms without it.

\section{Device Adapter Tag}
\label{sec:DeviceAdapterTag}

\index{device~adapter|(}

VTK-m uses a feature called a device adapter to define what type of device
will be used to run algorithms. The device adapter encapsulates the
device-specific code required to port to various devices. More information
on the function of the device adapter is given in
Section~\ref{sec:DeviceIndependence}.
% Link from original Dax document. Need to find a place for it.

The device adapter is identified by a \keyterm{device adapter tag}.
\index{device~adapter~tag} \index{tag!device~adapter} This tag, which is
simply an empty struct type, is used as the template parameter for several
classes in the VTK-m control environment and causes these classes to direct
their work to a particular device.

There are two ways to select a device adapter. The first is to make a
global selection of a default device adapter. The second is to specify a
specific device adapter as a template parameter.

\subsection{Default Device Adapter}

A default device adapter tag is specified in
\vtkmheader{vtkm/cont}{DeviceAdapter.h} (although it can also by specified
in many other VTK-m headers via header dependencies). If no other
information is given, VTK-m attempts to choose a default device adapter
that is a best fit for the system it is compiled on. VTK-m currently select
the default device adapter with the following sequence of conditions.

\fix{This is not currently the case in VTK-m, but should be once these
  adapters are implemented.}
\begin{itemize}
\item \index{CUDA} If the source code is being compiled by CUDA, the CUDA
  device is used.
\item \index{OpenMP} If the CUDA compiler is not being used and the current
  compiler supports OpenMP, then the OpenMP device is used.
\item \index{Intel Threading Building Blocks} \index{TBB} If the compiler
  supports neither CUDA nor OpenMP and VTK-m was configured to use Intel
  Threading Building Blocks, then that device is used.
\item \index{serial} If no parallel device adapters are found, then VTK-m
  falls back to a serial device.
\end{itemize}

You can also set the default device adapter specifically by setting the
\vtkmmacro{VTKM\_DEVICE\_ADAPTER} macro. This macro must be set
\emph{before} including any VTK-m files. You can set
\vtkmmacro{VTKM\_DEVICE\_ADAPTER} to any one of the following.

\fix{Most of these are not currently implemented.}
\begin{description}
\item[\vtkmmacro{VTKM\_DEVICE\_ADAPTER\_SERIAL}] Performs all computation on
  the same single thread as the control environment. This device is useful
  for debugging. This device is always available.
\item[\vtkmmacro{VTKM\_DEVICE\_ADAPTER\_CUDA}] Uses a CUDA capable GPU
  device. For this device to work, VTK-m must be configured to use CUDA and
  the code must be compiled by the CUDA \textfilename{nvcc} compiler.
\item[\vtkmmacro{VTKM\_DEVICE\_ADAPTER\_OPENMP}] Uses OpenMP compiler
  extensions to run algorithms on multiple threads. For this device to
  work, VTK-m must be configured to use OpenMP and the code must be
  compiled with a compiler that supports OpenMP pragmas.
\item[\vtkmmacro{VTKM\_DEVICE\_ADAPTER\_TBB}] Uses the Intel Threading
  Building Blocks library to run algorithms on multiple threads. For this
  device to work, VTK-m must be configured to use TBB and the executable
  must be linked to the TBB library.
\end{description}

These macros provide a useful mechanism for quickly reconfiguring code to
run on different devices. The following example shows a typical block of
code at the top of a source file that could be used for quick
reconfigurations.

\vtkmlisting{Macros to port VTK-m code among different devices}{DefaultDeviceAdapter.cxx}

The default device adapter can always be overridden by specifying a device
adapter tag, as described in the next section. There is actually one more
internal default device adapter named
\vtkmmacro{VTKM\_DEVICE\_ADAPTER\_ERROR} that will cause a compile error if
any component attempts to use the default device adapter. This feature is
most often used in testing code to check when device adapters should be
specified.

\subsection{Specifying Device Adapter Tags}

In addition to setting a global default device adapter, it is possible to
explicitly set which device adapter to use in any feature provided by
VTK-m. This is done by providing a device adapter tag as a template
argument to VTK-m templated objects. The following device adapter tags are
available in VTK-m.

\index{device~adapter~tag|(}
\index{tag!device~adapter|(}

\fix{Most of these are not currently implemented. The directories may change.}
\begin{description}
\item[\vtkmcont{DeviceAdapterTagSerial}] \index{serial} Performs all
  computation on the same single thread as the control environment. This
  device is useful for debugging. This device is always available. This tag
  is defined in \vtkmheader{vtkm/cont}{DeviceAdapterSerial.h}.
\item[\vtkmcudacont{DeviceAdapterTagCuda}] \index{CUDA} Uses a CUDA capable
  GPU device. For this device to work, VTK-m must be configured to use CUDA
  and the code must be compiled by the CUDA \textfilename{nvcc}
  compiler. This tag is defined in
  \vtkmheader{vtkm/cuda/cont}{DeviceAdapterCuda.h}.
\item[\vtkmopenmpcont{DeviceAdapterTagOpenMP}] \index{OpenMP} Uses OpenMP
  compiler extensions to run algorithms on multiple threads. For this
  device to work, VTK-m must be configured to use OpenMP and the code must be
  compiled with a compiler that supports OpenMP pragmas. This tag is
  defined in \vtkmheader{vtkm/openmp/cont}{DeviceAdapterOpenMP.h}.
\item[\vtkmtbbcont{DeviceAdapterTagTBB}]
  \index{Intel Threading Building Blocks} \index{TBB} Uses the Intel
  Threading Building Blocks library to run algorithms on multiple
  threads. For this device to work, VTK-m must be configured to use TBB and
  the executable must be linked to the TBB library. This tag is defined in
  \vtkmheader{vtkm/tbb/cont}{DeviceAdapterTBB.h}.
\end{description}

\fix{Add example. Not enough implemented to make a worthwhile example yet.}
%% The following example invokes the elevation worklet much like shown in
%% Example~\ref{ex:Elevation} on page~\pageref{ex:Elevation} but also
%% specifies using the Intel Threading Building blocks device adapter.
%% \index{Intel Threading Building Blocks} \index{TBB}
%% In particular, consider the template parameter of the
%% \daxcont{DispatcherMapField} class.
%% \begin{daxexample}{Calling the Elevation worklet with a specific device adapter.}
%% dax::worklet::Elevation elevation(dax::make_Vector3(-1.0, 0.0, 0.0),
%%                                   dax::make_Vector3(1.0, 0.0, 0.0),
%%                                   dax::make_Vector2(-1.0, 1.0));

%% dax::cont::DispatcherMapField<dax::worklet::Elevation, dax::tbb::cont::DeviceAdapterTagTBB>
%%     dispatcher(elevation);
%% dispatcher.Invoke(grid.GetPointCoordinates(), outPointElevation);
%% \end{daxexample}

When structuring your code to always specify a particular device adapter,
consider setting the default device adapter (with the
\vtkmmacro{VTKM\_DEVICE\_ADAPTER} macro) to
\vtkmmacro{VTKM\_DEVICE\_ADAPTER\_ERROR}. This will cause the compiler to
produce an error if any object is instantiated with the default device
adapter, which checks to make sure the code properly specifies every device
adapter parameter.

VTK-m also defines a macro named
\vtkmmacro{VTKM\_DEFAULT\_DEVICE\_ADAPTER\_TAG} that can be used in place
of an explicit device adapter tag to use the default tag. This macro is
used to create new templates that have template parameters for device
adapters that can use the default. The following example has a (rather
artificial) declaration of a helper class for executing the elevation
worklet.

\fix{Add example when worklets and dispatchers are implemented.}

%% \begin{daxexample}{Declaring a template with a default device adapter.}
%% template<typename DeviceAdapter = DAX_DEFAULT_DEVICE_ADAPTER_TAG>
%% class MyElevationDispatcher
%% {
%% public:
%%   void DoInvoke()
%%   {
%%     dax::cont::DispatcherMapField<dax::Worklet::Elevation,DeviceAdapter> dispatcher;
%%     dispatcher.Invoke(this->Grid.GetPointCoordinates(), this->OutPointElevation);
%% \end{daxexample}

\index{tag!device~adapter|)}
\index{device~adapter~tag|)}

\index{device~adapter|)}

\section{Array Handle}
\label{sec:ArrayHandle}

\index{array~handle|(}

An \keyterm{array handle}, implemented with the \vtkmcont{ArrayHandle}
class, manages an array of data that can be accessed or manipulated by VTK-m
algorithms. It is typical to construct an array handle in the control
environment to pass data to an algorithm running in the execution
environment. It is also typical for an algorithm running in the execution
environment to allocate and populate an array handle, which can then be
read back in the control environment. It is also possible for an array
handle to manage data created by one VTK-m algorithm and passed to another,
remaining in the execution environment the whole time and never copied to
the control environment.

The array handle may have up to two copies of the array, one for the
control environment and one for the execution environment. However,
depending on the device and how the array is being used, the array handle
will only have one copy when possible. Copies between the environments are
implicit and lazy. They are copied only when an operation needs data in an
environment where the data is not.

\vtkmcont{ArrayHandle} behaves like a shared smart pointer in that when the
C++ object is copied, each copy holds a reference to the same array. These
copies are reference counted so that when all copies of the
\vtkmcont{ArrayHandle} are destroyed, any allocated memory is released.

\subsection{Creating Array Handles}

\vtkmcont{ArrayHandle} is a templated class with two template
parameters. The first template parameter is the only one required and
specifies the base type of the entries in the array. The second template
parameter specifies the container used when storing data in the control
environment. Containers are discussed later in this section, and for now we
will use the default value.

\begin{vtkmexample}{Declaration of the \protect\vtkmcont{ArrayHandle} templated class.}
template<
    typename T,
    typename ArrayContainerControlTag = VTKM_DEFAULT_ARRAY_CONTAINER_CONTROL_TAG>
class ArrayHandle;
\end{vtkmexample}

There are multiple ways to create and populate an array handle. The default
\vtkmcont{ArrayHandle} constructor will create an empty array with nothing
allocated in either the control or execution environment. This is
convenient for creating arrays used as the output for algorithms.

\vtkmlisting{Creating an \textidentifier{ArrayHandle} for output data.}{CreateArrayHandle.cxx}

Constructing an \vtkmcont{ArrayHandle} that points to a provided C array or
\textcode{std::vector} is straightforward with the
\vtkmcont{make\_ArrayHandle} functions. These functions will make an array
handle that points to the array data that you provide.

\vtkmlisting{Creating an \textidentifier{ArrayHandle} that points to a provided C array.}{ArrayHandleFromCArray.cxx}

\vtkmlisting[ex:ArrayHandleFromVector]{Creating an \textidentifier{ArrayHandle} that points to a provided \textcode{std::vector}.}{ArrayHandleFromVector.cxx}

\emph{Be aware} that \vtkmcont{make\_ArrayHandle} makes a shallow pointer
copy. This means that if you change or delete the data provided, the
internal state of \vtkmcont{ArrayHandle} becomes invalid and undefined
behavior can ensue. The most common manifestation of this error happens
when a \textcode{std::vector} goes out of scope. This subtle interaction
will cause the \vtkmcont{ArrayHandle} to point to an unallocated portion of
the memory heap. For example, if the code in
Example~\ref{ex:ArrayHandleFromVector} where to be placed within a callable
function or method, it could cause the \vtkmcont{ArrayHandle} to become
invalid.

\vtkmlisting{Invalidating an \textidentifier{ArrayHandle} by letting the source \textcode{std::vector} leave scope.}{ArrayOutOfScope.cxx}

\subsection{Array Portals}
\label{sec:ArrayPortals}

\index{array~portal|(}

An array handle defines auxiliary structures called \keyterm{array portals}
that provide direct access into its data. An array portal is a simple
object that is somewhat functionally equivalent to an STL-type iterator, but
with a much simpler interface. Array portals can be read-only (const) or
read-write and they can be accessible from either the control environment
or the execution environment. All these variants have similar interfaces
although some features that are not applicable can be left out.

An array portal object contains each of the following:
\begin{description}
\item[\textcode{ValueType}] A \textcode{typedef} of the type for each item
  in the array.
\item[\textcode{GetNumberOfValues}] A method that returns the number of
  entries in the array.
\item[\textcode{Get}] A method that returns the value at a given index.
\item[\textcode{Set}] A method that changes the value at a given
  index. This method does not need to exist for read-only (const) array
  portals.
\item[\textcode{IteratorType}] A \textcode{typedef} of an STL-compatible
  random-access iterator that can be used for alternative access. This
  method does not need to exist in the execution environment.
\item[\textcode{GetIteratorBegin}] A method that returns an STL-compatible
  iterator of type \textcode{IteratorType} that points to the beginning of
  the array. This method does not need exist in the execution environment.
\item[\textcode{GetIteratorEnd}] A method that returns an STL-compatible
  iterator of type \textcode{IteratorType} that points to the beginning of
  the array. This method does not need to exist in the execution
  environment.
\end{description}

The following code example defines an array portal for a simple C array of
scalar values. This definition has no practical value (it is covered by the
more general \vtkmcontinternal{ArrayPortalFromIterators}), but demonstrates
the function of each component.

\vtkmlisting{A simple array portal implementation.}{SimpleArrayPortal.cxx}

\vtkmcont{ArrayHandle} contains two \textcode{typedef}s for array portal
types that are capable of interfacing with the underlying data in the
control environment. These are \textcode{PortalControl}
\index{PortalControl} and \textcode{PortalConstControl},
\index{PortalConstControl} which define read-write and read-only (const)
array portals, respectively.

\vtkmcont{ArrayHandle} also contains similar \textcode{typedef}s for array
portals in the execution environment. Because these types are dependent on
the device adapter used for execution, these typedefs are embedded in a
templated class named \textcode{ExecutionTypes}. \index{ExecutionTypes}
Within \textcode{ExecutionTypes} are the typedefs \textcode{Portal} and
\textcode{PortalConst} defining the read-write and read-only (const) array
portals, respectively, for the execution environment for the given device
adapter tag.

Because \vtkmcont{ArrayHandle} is an control environment object, it
provides the methods \textcode{GetPortalControl} \index{GetPortalControl}
and \textcode{GetPortalConstControl} \index{GetPortalConstControl} to get
the associated array portal objects. These methods also have the side
effect of refreshing the control environment copy of the data, so this can
be a way of synchronizing the data. Be aware that when an
\vtkmcont{ArrayHandle} is created with a pointer or \textcode{std::vector},
it is put in a read-only mode, and \textcode{GetPortalControl} can fail
(although \textcode{GetPortalConstControl} will still work). Also be aware
that calling \textcode{GetPortalControl} will invalidate any copy in the
execution environment, meaning that any subsequent use will cause the data
to be copied back again.

\vtkmlisting{Using portals from an \textidentifier{ArrayHandle}.}{ControlPortals.cxx}

\index{array~portal|)}

\subsection{Interface to Execution Environment}
\label{sec:ArrayHandleInterfaceToExecutionEnvironment}

One of the main functions of the array handle is to allow an array to be
defined in the control environment and then be used in the execution
environment. When using an \textidentifier{ArrayHandle} with worklets
\fix{or filters?}, this transition is handled automatically. However, it is
also possible to invoke the transfer for use in your own scheduled
algorithms.

The \textidentifier{ArrayHandle} class manages the transition from control
to execution with a set of three methods that allocate, transfer, and ready
the data in one operation. These methods all start with the prefix
\textcode{Prepare} and are meant to be called before some operation happens
in the execution environment. The methods are as follows.

\begin{description}
\item[\textcode{PrepareForInput}] \index{PrepareForInput} Copies data from
  the control to the execution environment, if necessary, and readies the
  data for read-only access.
\item[\textcode{PrepareForInPlace}] \index{PrepareForInPlace} Copies the
  data from the control to the execution environment, if necessary, and
  readies the data for both reading and writing.
\item[\textcode{PrepareForOutput}] \index{PrepareForOutput} Allocates space
  (the size of which is given as a parameter) in the execution environment,
  if necessary, and readies the space for writing.
\end{description}

Each of these methods takes a single argument that is the device adapter
tag for the device to run the execution environment in (see
Section~\ref{sec:DeviceAdapterTag} for more information on device adapter
tags).  Each of these methods returns an array portal that can be used in
the execution environment. \textcode{PrepareForInput} returns an object of
type
\textidentifier{ArrayHandle}\textcode{:\colonhyp{}ExecutionTypes<{\it{}DeviceAdapterTag}>:\colonhyp{}PortalConst}
whereas \textcode{PrepareForInPlace} and \textcode{PrepareForOutput} each
return an object of type
\textidentifier{ArrayHandle}\textcode{:\colonhyp{}ExecutionTypes<{\it{}DeviceAdapterTag}>:\colonhyp{}Portal}.

Although these \textcode{Prepare} methods are called in the control
environment, the returned array portal can only be used in the execution
environment. Thus, the portal must be passed to an invocation of the
execution environment. Typically this is done with a call to
\textcode{Schedule} in \vtkmcont{DeviceAdapterAlgorithm}. This and other
device adapter algorithms are described in detail in
Section~\ref{sec:DeviceAdapterAlgorithms}, but here is a quick example of
using these execution array portals in a simple functor.

\vtkmlisting{Using an execution array portal from an \textidentifier{ArrayHandle}.}{ExecutionPortals.cxx}

It should be noted that the array handle will expect any use of the
execution array portal to finish before the next call to any
\textidentifier{ArrayHandle} method. Since these \textcode{Prepare} methods
are typically used in the process of scheduling an algorithm in the
execution environment, this is seldom an issue.

\subsection{Basic Container}

\index{array~handle!container|(}
\index{container|(}

As previously discussed, \vtkmcont{ArrayHandle} takes two template arguments.
\begin{vtkmexample}{Declaration of the \protect\vtkmcont{ArrayHandle} templated class (again).}
template<
    typename T,
    typename ArrayContainerControlTag = VTKM_DEFAULT_ARRAY_CONTAINER_CONTROL_TAG>
class ArrayHandle;
\end{vtkmexample}
The first argument is the only one required and has been demonstrated
multiple times before. The second (optional) argument specifies something
called a container, which provides the interface between the generic
\vtkmcont{ArrayHandle} class and a specific storage mechanism in the
control environment.

In this and the following sections we describe these control environment
containers. A default container is specified in much the same way as a
default device adapter is defined. It is done by setting the
\vtkmmacro{VTKM\_ARRAY\_CONTAINER\_CONTROL} macro. This macro must be set
before including any VTK-m header files. Currently the only practical
container provided by the VTK-m toolkit is the basic container, which simply
allocates a continuous section of memory of the given base type. This
container can be explicitly specified by setting
\vtkmmacro{VTKM\_ARRAY\_CONTAINER\_CONTROL} to
\vtkmmacro{VTKM\_ARRAY\_CONTAINER\_CONTROL\_BASIC} although the basic
container will also be used as the default if no other container is
specified (which is typical).

The default array container can always be overridden by specifying an array
container tag. The tag for the basic container is located in the
\vtkmheader{vtkm/cont}{ArrayContainerControl.h} header file and is named
\vtkmcont{ArrayContainerControlTagBasic}. Here is an example of specifying
the container type when declaring an array handle.

\vtkmlisting{Specifying the container type for an \textidentifier{ArrayHandle.}}{ArrayHandleContainerParameter.cxx}

VTK-m also defines a macro named
\vtkmmacro{VTKM\_DEFAULT\_ARRAY\_CONTAINER\_CONTROL\_TAG} that can be used
in place of an explicit array container tag to use the default tag. This
macro is used to create new templates that have template parameters for
array containers that can use the default or to create array handles with
the default container but a specific device adapter.

\subsection{Adapting Data Structures}

\index{array~handle!adapting|(}
\index{container!adapting|(}

The intention of the container parameter for \vtkmcont{ArrayHandle} is to
implement the strategy design pattern to enable VTK-m to interface directly
with the data of any third party code source. VTK-m is designed to work
with data originating in other libraries or applications. By creating a new
type of array container, VTK-m can be entirely adapted to new kinds of data
structures.

In this section we demonstrate the steps required to adapt the array handle
to a data structure provided by a third party. For the purposes of the
example, let us say that some fictitious library named ``foo'' has a simple
structure named \textcode{FooFields} that holds the field values for a
particular part of a mesh, and then maintain the field values for all
locations in a mesh in a \textcode{std::deque} object.

\vtkmlisting{Fictitious field storage used in custom array container examples.}{FictitiousFieldStorage.cxx}

VTK-m expects separate arrays for each of the fields rather than a single
array containing a structure holding all of the fields. However, rather
than copy each field to its own array, we can create a container for each
field that points directly to the data in a \textcode{FooFieldsDeque}
object.

The first step in creating an adapter container is to create a control
environment array portal to the data. This is described in more detail in
Section~\ref{sec:ArrayPortals} and is generally straightforward for simple
containers like this. Here is an example implementation for our
\textcode{FooFieldsDeque} container.

\vtkmlisting[ex:ArrayPortalAdapter]{Array portal to adapt a third-party container to VTK-m.}{ArrayPortalAdapter.cxx}

The next step in creating an adapter container is to define a tag for the
adapter. We shall call ours
\textcode{ArrayContainerControlTagFooPressure}. Then, we need to create a
specialization of the templated \vtkmcontinternal{ArrayContainerControl}
class. The \textidentifier{ArrayHandle} will instantiate an object using
the array container tag we give it, and we define our own specialization so
that it runs our interface into the code.

\vtkmcontinternal{ArrayContainerControl} has two template arguments: the
base type of the array and the array container tag.

\vtkmlisting{Prototype for \protect\vtkmcontinternal{ArrayContainerControl}.}{ArrayContainerControlPrototype.cxx}

The \vtkmcontinternal{ArrayContainerControl} must define the following items.
\begin{description}
\item[\textcode{ValueType}] A \textcode{typedef} of the type for each item
  in the array. This is the same type as the first template argument.
\item[\textcode{PortalType}] The type of an array portal that can be used
  to access the underlying data. This array portal needs to work only in
  the control environment.
\item[\textcode{PortalConstType}] A read-only (const) version of
  \textcode{PortalType}.
\item[\textcode{GetPortal}] A method that returns an array portal of type
  \textcode{PortalType} that can be used to access the data manged in this
  container.
\item[\textcode{GetPortalConst}] Same as \textcode{GetPortal} except it
  returns a read-only (const) array portal.
\item[\textcode{GetNumberOfValues}] A method that returns the number of
  values the container is currently allocated for.
\item[\textcode{Allocate}] A method that allocates the array to a given
  size. An values stored in the previous allocation may be destroyed.
\item[\textcode{Shrink}] A method like \textcode{Allocate} with two
  differences. First, the size of the allocation must be smaller than the
  existing allocation when the method is called. Second, any values
  currently stored in the array will be valid after the array is
  resized. This constrained form of allocation allows the array to be
  resized and values valid without ever having to copy data.
\item[\textcode{ReleaseResources}] A method that instructs the container to
  free all of its memory.
\end{description}

The following provides an example implementation of our adapter to a
\textcode{FooFieldsDeque}. It relies on the
\textcode{ArrayPortalFooPressure} provided in
Example~\ref{ex:ArrayPortalAdapter}.

\vtkmlisting{Array container to adapt a third-party container to VTK-m.}{ArrayContainerControlAdapter.cxx}

The final step to make a container adapter is to make a mechanism to
construct an \textidentifier{ArrayHandle} that points to a particular
container. This can be done by creating a trivial subclass of
\vtkmcont{ArrayHandle} that simply constructs the array handle to the state
of an existing container.

\vtkmlisting{Array handle to adapt a third-party container to VTK-m.}{ArrayHandleAdapter.cxx}

%% With this new version of \textidentifier{ArrayHandle}, the Dax toolkit can
%% now read to and write from the \textcode{FooFieldsDeque} structure
%% directly. Note, however, that when writing to an array handle, it is
%% necessary to call \textcode{GetPortalControl} or
%% \textcode{GetPortalConstControl} to flush data from the execution
%% environment to the control environment. \fix{Should probably make this
%%   easier.}

%% \begin{daxexample}{Using an \textidentifier{ArrayHandle} with custom container.}
%% template<typename GridType>
%% DAX_CONT_EXPORT
%% void GetElevationAirPressure(const GridType &grid, FooFieldsDeque *fields)
%% {
%%   dax::worklet::Elevation elevation(dax::make_Vector3(0.0, 0.0, 0.0),
%%                                     dax::make_Vector3(0.0, 0.0, 10.0),
%%                                     dax::make_Vector2(0.02, 0.0));

%%   // Make an array handle that points to the pressure values in fields.
%%   ArrayHandleFooPressure pressureHandle(fields);

%%   // Run the elevation worklet.
%%   dax::cont::DispatcherMapField<dax::worklet::Elevation> dispatcher(elevation);
%%   dispatcher.Invoke(grid.GetPointCoordinates(), pressureHandle);

%%   // Make sure values are flushed back to the control environment.
%%   pressureHandle.GetPortalConstControl();

%%   // Now the pressure fields are field in the fields container.
%% };
%% \end{daxexample}

\index{container!adapting|)}
\index{array~handle!adapting|)}

%% \subsection{Implicit Containers}

%% \index{array~handle!implicit|(}
%% \index{container!implicit|(}
%% \index{implicit~container|(}
%% \index{functional~array|(}

%% The generic array handle and array container templating in the Dax toolkit
%% allows for any type of operations to retrieve a particular value. Typically
%% this is used to convert an index to some location or locations in
%% memory. However, it is also possible to compute a value directly from an
%% index rather than look up some value in memory. Such an array is completely
%% functionally and requires no storage in memory at all. Such a functional
%% array is specified with an \keyterm{implicit container}

%% Specifying a functional or implicit array in the Dax toolkit is
%% straightforward. The Dax toolkit comes with a generic implicit container
%% that can be templated to any function you like. In this section we
%% demonstrate the steps required to create an implicit container. For the
%% purposes of the example, let us say we want an array of even numbers. That
%% is, the array has the values $[0,2,4,6,\ldots]$ up to some given
%% size. Although we could easily create this array in memory, we can save
%% space and possibly time by computing these values on demand.

%% The first step to creating an implicit container is to build a read-only
%% array portal that computes the desired value in the \textcode{Get}
%% method. The portal must work in both the control and execution environments
%% (although the iterators only need to work in the control environment), and
%% no \textcode{Set} method is necessary because the array is assumed to be
%% read-only (since it is functional). The array portal may have a small
%% amount of state, but the class itself must be copyable as a raw data
%% structure. That is, using \textcode{memcpy} on the structure should work.

%% \begin{daxexample}[ex:ImplicitArrayPortal]{Implicit array portal for an implicit array of even numbers.}
%% #include <dax/cont/ArrayContainerControlImplicit.h>
%% #include <dax/cont/ArrayHandle.h>
%% #include <dax/cont/internal/IteratorFromArrayPortal.h>

%% class ArrayPortalEvenNumbers
%% {
%% public:
%%   typedef dax::Id ValueType;

%%   DAX_EXEC_CONT_EXPORT
%%   ArrayPortalEvenNumbers() : NumberOfValues(0) {  }

%%   DAX_EXEC_CONT_EXPORT
%%   ArrayPortalEvenNumbers(dax::Id numValues) : NumberOfValues(numValues) {  }

%%   DAX_EXEC_CONT_EXPORT
%%   dax::Id GetNumberOfValues() const { return this->NumberOfValues; }

%%   DAX_EXEC_CONT_EXPORT
%%   ValueType Get(dax::Id index) const { return 2*index; }

%%   typedef dax::cont::internal::IteratorFromArrayPortal<ArrayPortalEvenNumbers> IteratorType;

%%   DAX_CONT_EXPORT
%%   IteratorType GetIteratorBegin() const
%%   {
%%     return IteratorType(*this);
%%   }

%%   DAX_CONT_EXPORT
%%   IteratorType GetIteratorEnd() const
%%   {
%%     return IteratorType(*this, this->NumberOfValues);
%%   }

%% private:
%%   dax::Id NumberOfValues;
%% };
%% \end{daxexample}

%% Note that this array portal uses the template
%% \daxcontinternal{IteratorFromArrayPortal}, which can convert any array
%% portal to STL-compatible iterators.

%% Once the implicit array portal is built, an implicit array container is
%% defined using the \daxcont{ArrayContainerControlTagImplicit} tag. This tag
%% is templated, and the template parameter is the implicit array portal.

%% \begin{daxexample}[ex:ImplicitArrayContainer]{Defining the container tag for an implicit array of even numbers.}
%% typedef dax::cont::ArrayContainerControlTagImplicit<ArrayPortalEvenNumbers>
%%     ArrayContainerControlTagEvenNumbers;
%% \end{daxexample}

%% An array handle can be created directly with this tag as the container
%% template parameter to \daxcont{ArrayHandle}. However, it is common to
%% create a trivial subclass of \daxcont{ArrayHandle} that simply constructs
%% the array handle to an implicit array portal of a given state. \fix{There
%%   remains an issue with the concept template matching mechanism that can
%%   cause this subclass to fail with these ArrayHandle subclasses.} The
%% following example, which builds on Examples \ref{ex:ImplicitArrayPortal}
%% and \ref{ex:ImplicitArrayContainer} demonstrates the convenience
%% \daxcont{ArrayHandle} subclass.

%% \begin{daxexample}{Implicit array handle of even numbers.}
%% template<typename DeviceAdapter>
%% class ArrayHandleEvenNumbers
%%     : public dax::cont::ArrayHandle<
%%                 dax::Id, ArrayContainerControlTagEvenNumbers, DeviceAdapter>
%% {
%%   typedef dax::cont::ArrayHandle<
%%                 dax::Id, ArrayContainerControlTagEvenNumbers, DeviceAdapter> Superclass;

%% public:
%%   ArrayHandleEvenNumbers(dax::Id length)
%%     : Superclass(ArrayPortalEvenNumbers(length)) {  }
%% };
%% \end{daxexample}

%% The Dax toolkit comes with two examples of implicit containers. The first
%% is \daxcont{ArrayHandleConstant}, which returns the same value for every
%% index in the array. The constant array is useful when an algorithm that can
%% work on a variable field is used on a constant value. The second is
%% \daxcont{ArrayHandleCounting}, which returns the index as the value with a
%% possible offset. The counting array is useful for generating fields of
%% identifiers or for indexing operations. The Dax toolkit also provides
%% \daxcont{make\_ArrayHandleConstant} and \daxcont{make\_ArrayHandleCounting}
%% convenience functions to simplify building these arrays.

%% \index{functional~array|)}
%% \index{implicit~container|)}
%% \index{container!implicit|)}
%% \index{array~handle!implicit|)}

%% \subsection{Derived Containers}

%% \index{array~handle!derived|(}
%% \index{container!derived|(}

%% So far, we have discussed using the array container mechanism to adapt to
%% particular memory layout and to create implicit arrays. Yet another option
%% is to create a \keyterm{derived container}. A derived container shares
%% attributes with both adaptive containers and implicit containers. A derived
%% container takes one or more other arrays and changes their behavior in some
%% way. Their implementation is similar to adapting a memory layout, but some
%% of the details are different.

%% In this section we will demonstrate the steps required to create a derived
%% container. For the purposes of the example, let us say we want to array
%% handles to behave as one array with the contents concatenated together. We
%% could of course actually copy the data, but we can also do it in place.

%% As before, the first step to creating a derived container is to build an
%% array portal that will take portals from arrays being derived. The portal
%% must work in both the control and execution environment (or have a separate
%% version for control and execution).

%% \begin{daxexample}[ex:DerivedArrayPortal]{Derived array portal for concatenated arrays.}
%% #include <dax/cont/ArrayContainerControlImplicit.h>
%% #include <dax/cont/ArrayPortal.h>
%% #include <dax/cont/Assert.h>
%% #include <dax/cont/internal/IteratorFromArrayPortal.h>

%% template<template P1, template P2>
%% class ArrayPortalConcatenate
%% {
%% public:
%%   typedef P1 PortalType1;
%%   typedef P2 PortalType2;
%%   typedef typename PortalType1::ValueType ValueType;

%%   DAX_EXEC_CONT_EXPORT
%%   ArrayPortalConcatenate() : FirstPortal(), Portal2() {   }

%%   DAX_EXEC_CONT_EXPORT
%%   ArrayPortalConcatenate(const PortalType1 &firstPortal,
%%                          const PortalType2 &secondPortal)
%%     : Portal1(firstPortal), Portal2(secondPortal) {  }

%%   /// Copy constructor for any other ArrayPortalConcatenate with an iterator
%%   /// type that can be copied to this iterator type. This allows us to do any
%%   /// type casting that the iterators do (like the non-const to const cast).
%%   template<class OtherP1, class OtherP2>
%%   DAX_CONT_EXPORT
%%   ArrayPortalConcatenate(const ArrayPortalConcatenate<OtherP1,OtherP2> &src)
%%     : Portal1(src.GetPortal1()), Portal2(src.GetPortal2()) {  }

%%   DAX_EXEC_CONT_EXPORT
%%   dax::Id GetNumberOfValues() const {
%%     return this->Portal1.GetNumberOfValues() + this->Portal2.GetNumberOfValues();
%%   }

%%   DAX_EXEC_CONT_EXPORT
%%   ValueType Get(dax::Id index) const {
%%     if (index < this->Portal1.GetNumberOfValues())
%%       {
%%       return this->Portal1.Get(index);
%%       }
%%     else
%%       {
%%       return this->Portal2.Get(index);
%%       }
%%   }

%%   DAX_EXEC_CONT_EXPORT
%%   ValueType Set(dax::Id index, const ValueType &value) const {
%%     if (index < this->Portal1.GetNumberOfValues())
%%       {
%%       return this->Portal1.Set(index, value);
%%       }
%%     else
%%       {
%%       return this->Portal2.Set(index, value);
%%       }
%%   }

%%   typedef dax::cont::internal::IteratorFromArrayPortal<
%%       ArrayPortalConcatenate<PortalType1,PortalType2> > IteratorType;

%%   DAX_CONT_EXPORT
%%   IteratorType GetIteratorBegin() const {
%%     return IteratorType(*this);
%%   }

%%   DAX_CONT_EXPORT
%%   IteratorType GetIteratorEnd() const {
%%     return IteratorType(*this, this->GetNumberOfValues());
%%   }

%%   DAX_EXEC_CONT_EXPORT
%%   const PortalType1 &GetPortal1() const { return this->Portal1; }
%%   DAX_EXEC_CONT_EXPORT
%%   const PortalType2 &GetPortal2() const { return this->Portal2; }

%% private:
%%   PortalType1 Portal1;
%%   PortalType2 Portal2;
%% };
%% \end{daxexample}

%% Like in an adapter container, the next step in creating a derived container
%% is to define a tag for the adapter. We shall call ours
%% \textcode{ArrayContainerControlTagConcatenate} and it will be templated on
%% the two array handle types that we are deriving. Then, we need to create a
%% specialization of the templated \daxcontinternal{ArrayContainerControl}
%% class. The implementation for an \textidentifier{ArrayContainerControl} for
%% a derived container is usually trivial compared to an adapter container
%% because the majority of the work is deferred to the derived arrays.

%% \begin{daxexample}[ex:DerivedArrayContainer]{\textidentifier{ArrayContainerControl} for derived container of concatenated arrays.}
%% template<typename ArrayHandleType1, typename ArrayHandleType2>
%% struct ArrayContainerControlTagConcatenate {  };

%% namespace dax {
%% namespace cont {
%% namespace internal {

%% template<typename T, typename Container1, typename Container2, typename DeviceAdapter>
%% class ArrayContainerControl<
%%     T,
%%     ArrayContainerControlTagConcatenate<
%%         dax::cont::ArrayHandle<T, Container1, DeviceAdapter>
%%         dax::cont::ArrayHandle<T, Container2, DeviceAdapter> > >
%% {
%%   typedef dax::cont::ArrayHandle<T, Container1, DeviceAdapter> ArrayHandleType1;
%%   typedef dax::cont::ArrayHandle<T, Container2, DeviceAdapter> ArrayHandleType2;

%% public:
%%   typedef T ValueType;

%%   typedef ArrayPortalConcatinate<
%%       typename ArrayHandleType1::PortalControl,
%%       typename ArrayHandleType2::PortalControl> PortalType;
%%   typedef ArrayPortalConcatinate<
%%       typename ArrayHandleType1::PortalConstControl,
%%       typename ArrayHandleType2::PortalConstControl> PortalConstType;

%%   DAX_CONT_EXPORT
%%   ArrayContainerControl() : Valid(false) {  }

%%   DAX_CONT_EXPORT
%%   ArrayContainerControl(const ArrayHandleType1 firstArrayHandle,
%%                         const ArrayHandle2 secondArrayHandle)
%%     : Array1(firstArrayHandle), Array2(secondArrayHandle) {  }

%%   DAX_CONT_EXPORT
%%   PortalType GetPortal() {
%%     DAX_ASSERT_CONT(this->Valid);
%%     return PortalType(this->Array1.GetPortalControl(), this->Array2.GetPortalControl());
%%   }

%%   DAX_CONT_EXPORT
%%   PortalConstType GetPortalConst() const {
%%     DAX_ASSERT_CONT(this->Valid);
%%     return PortalType(this->Array1.GetPortalConstControl(),
%%                       this->Array2.GetPortalConstControl());
%%   }

%%   DAX_CONT_EXPORT
%%   dax::Id GetNumberOfValues() const {
%%     DAX_ASSERT_CONT(this->Valid);
%%     return this->Array1.GetNumberOfValues() + this->Array2.GetNumberOfValues();
%%   }

%%   DAX_CONT_EXPORT
%%   void Allocate(dax::Id numberOfValues) {
%%     DAX_ASSERT_CONT(this->Valid);
%%     // This implementation of allocate, which allocates the same amount in both arrays, is
%%     // arbitrary. It could, for example, leave the size of Array1 alone and change the size
%%     // of Array2. Or, probably most likely, it could simply throw an error and state that
%%     // this operation is invalid.
%%     dax::Id half = numberOfValues/2;
%%     // PrepareForOutput is the only accessible way to resize an ArrayHandle.
%%     this->Array1.PrepareForOutput(numberOfValues-half);
%%     this->Array2.PrepareForOutput(half);
%%   }

%%   DAX_CONT_EXPORT
%%   void Shrink(dax::Id numberOfValues) {
%%     DAX_ASSERT_CONT(this->Valid);
%%     if (numberOfValues < this->Array1.GetNumberOfValues())
%%       {
%%       this->Array1.Shrink(numberOfValues);
%%       this->Array2.Shrink(0);
%%       }
%%     else
%%       {
%%       this->Array2.Shrink(numberOfValues - this->Array1.GetNumberOfValues());
%%       }
%%   }

%%   DAX_CONT_EXPORT
%%   void ReleaseResources() {
%%     DAX_ASSERT_CONT(this->Valid);
%%     this->Array1.ReleaseResources();
%%     this->Array2.ReleaseResources();
%%   }

%% private:
%%   ArrayHandleType1 Array1;
%%   ArrayHandleType2 Array2;
%%   bool Valid;
%% };

%% }
%% }
%% } // namespace dax::cont::internal
%% \end{daxexample}

%% One of the responsibilities of an array handle is to copy data between the
%% control and execution environments. The default behavior is to request the
%% device adapter to copy data items from one environment to another. This
%% might involve transferring data between a host and device. For an array of
%% data resting in memory, this is necessary. However, implicit containers
%% (described in the previous section) override this behavior to pass nothing
%% but the functional array portal. Likewise, it is undesirable to do a raw
%% transfer of data with derived containers. The underlying arrays being
%% derived may be used in other contexts, and it would be good to share the
%% data wherever possible. It is also sometimes more efficient to copy data
%% independently from the arrays being derived than from the derived container
%% itself.

%% \index{array~transfer|(}

%% The mechanism that controls how a particular control array container gets
%% transferred to and from the execution environment is encapsulated in the
%% templated \daxcontinternal{ArrayTransfer} class. By creating a
%% specialization of \daxcontinternal{ArrayTransfer}, we can modify the
%% transfer behavior to instead transfer the arrays being derived and use the
%% respective copies in the control and execution environments.

%% \daxcontinternal{ArrayTransfer} has three template arguments: the base type
%% of the array, the array container tag, and the device adapter tag.

%% \begin{daxexample}{Prototype for \protect\daxcontinternal{ArrayTransfer}.}
%% namespace dax {
%% namespace cont {
%% namespace internal {

%% template<typename T, class ArrayContainerControlTag, class DeviceAdapterTag>
%% class ArrayTransfer;

%% }
%% }
%% }
%% \end{daxexample}

%% The \daxcontinternal{ArrayTransfer} must define the following items.
%% \begin{description}
%% \item[\textcode{ValueType}] A \textcode{typedef} of the type for each item
%%   in the array. This is the same type as the first template argument.
%% \item[\textcode{PortalControl}] The type of an array portal that is used to
%%   access the underlying data in the control environment.
%% \item[\textcode{PortalConstControl}] A read-only (const) version of
%%   \textcode{PortalControl}.
%% \item[\textcode{PortalExecution}] The type of an array portal that is used
%%   to access the underlying data in the execution environment.
%% \item[\textcode{PortalConstExecution}] A read-only (const) version of
%%   \textcode{PortalExecution}.
%% \item[\textcode{GetNumberOfValues}] A method that returns the number of
%%   values currently allocated in the execution environment. The results may
%%   be undefined if none of the load or allocate methods have yet been
%%   called.
%% \item[\textcode{LoadDataForInput}] A method that takes an array portal of
%%   type \textcode{PortalConstControl}, allocates enough space in the
%%   execution environment, and copies the given data to that array. The
%%   allocated array can later be accessed via the
%%   \textcode{GetPortalConstExecution} method. The data is assumed to be
%%   read-only.
%% \item[\textcode{LoadDataForInPlace}] A method that takes an array portal of
%%   type \textcode{PortalControl}, allocates enough space in the execution
%%   environment, and copies the given data to that array. The allocated array
%%   can later be accessed via the \textcode{GetPortalExecution} and
%%   \textcode{GetPortalConstExecution} methods. The data can be read and
%%   written.
%% \item[\textcode{AllocateArrayForOutput}] A method that takes an array
%%   container and a size and allocates an array in the execution environment
%%   of the specified size. The initial memory is uninitialized and can be
%%   accessed via the \textcode{GetPortalExecution} method. The container
%%   argument can be used to allocate data when the control and execution
%%   share arrays, but this argument is often ignored.
%% \item[\textcode{RetrieveOutputData}] This method takes an array container,
%%   allocates memory in the control environment, and copies data from the
%%   execution environment into it.
%% \item[\textcode{CopyInto}] This method takes an STL-compatible iterator and
%%   copies data from the execution environment into it.
%% \item[\textcode{Shrink}] A method that adjusts the size of the array in the
%%   execution environment to something that is a smaller size. All the data
%%   up to the new length must remain valid. Typically, no memory is actually
%%   reallocated. Instead, a different end is marked.
%% \item[\textcode{GetPortalExecution}] A method that returns an array portal
%%   that can be used in the execution environment. The portal was defined in
%%   either \textcode{LoadDataForInPlace} or
%%   \textcode{AllocateArrayForOutput}.
%% \item[\textcode{GetPortalConstExecution}] A method that returns a read-only
%%   (const) array portal that can be used in the execution environment. The
%%   portal was defined in one of the load or allocate methods.
%% \item[\textcode{ReleaseResources}] A method that frees any resources
%%   (typically memory) in the execution environment.
%% \end{description}

%% Continuing our example derived container that concatenates two arrays
%% started in Examples \ref{ex:DerivedArrayPortal} and
%% \ref{ex:DerivedArrayContainer}, the following provides an
%% \textidentifier{ArrayTransfer} appropriate for the derived container.

%% \begin{daxexample}[ex:DerivedArrayTransfer]{\textidentifier{ArrayTransfer} for derived container of concatenated arrays.}
%% namespace dax {
%% namespace cont {
%% namespace internal {

%% template<class ArrayHandleType1,
%%          class ArrayHandleType2,
%%          class DeviceAdapter>
%% class ArrayTransfer<
%%     typename ArrayHandleType1::ValueType,
%%     ArrayContainerControlTagConcatenate<ArrayHandleType1,ArrayHandleType2>,
%%     DeviceAdapter>
%% {
%% public:
%%   typedef typename ArrayHandleType1::ValueType ValueType;

%% private:
%%   typedef
%%   ArrayContainerControlTagConcatenate<ArrayHandleType1,ArrayHandleType2>
%%     ContainerTag;
%%   typedef dax::cont::internal::ArrayContainerControl<ValueType, ContainerTag> ContainerType;

%% public:
%%   typedef typename ContainerType::PortalType PortalControl;
%%   typedef typename ContainerType::PortalConstType PortalConstControl;

%%   typedef ArrayPortalConcatinate<
%%       typename ArrayHandleType1::PortalExecution,
%%       typename ArrayHandleType2::PortalExecution> PortalExecution;
%%   typedef ArrayPortalConcatinate<
%%       typename ArrayHandleType1::PortalConstExecution,
%%       typename ArrayHandleType2::PortalConstExecution> PortalConstExecution;

%%   DAX_CONT_EXPORT
%%   ArrayTransfer()
%%     : ArraysValid(false),
%%       ExecutionPortalConstValid(false),
%%       ExecutionPortalValid(false)
%%   {  }

%%   DAX_CONT_EXPORT
%%   ArrayTransfer(ArrayHandleType1 firstArray,
%%                 ArrayHandleType2 secondArray)
%%     : Array1(firstArray),
%%       Array2(secondArray),
%%       ArraysValid(true),
%%       ExecutionPortalConstValid(false),
%%       ExecutionPortalValid(false)
%%   {  }

%%   DAX_CONT_EXPORT
%%   dax::Id GetNumberOfValues() const {
%%     DAX_ASSERT_CONT(this->ArraysValid);
%%     return this->Array1.GetNumberOfValues() + this->Array2.GetNumberOfValues();
%%   }

%%   DAX_CONT_EXPORT
%%   void LoadDataForInput(PortalConstControl daxNotUsed(portal)) {
%%     // Assuming portal was created from a container with the same two arrays.
%%     DAX_ASSERT_CONT(this->ArraysValid);
%%     this->ExecutionPortalConst = PortalConstExecution(this->Array1.PrepareForInput(),
%%                                                       this->Array2.PrepareForInput());
%%     this->ExecutionPortalConstValid = true;
%%     this->ExecutionPortalValid = false;
%%   }

%%   DAX_CONT_EXPORT
%%   void LoadDataForInPlace(PortalControl daxNotUsed(portal)) {
%%     // Assuming portal was created from a container with the same two arrays.
%%     DAX_ASSERT_CONT(this->ArraysValid);
%%     this->ExecutionPortal = PortalExecution(this->Array1.PrepareForInPlace(),
%%                                             this->Array2.PrepareForInPlace());

%%     this->ExecutionPortalConst = this->ExecutionPortal;
%%     this->ExecutionPortalConstValid = true;
%%     this->ExecutionPortalValid = true;
%%   }

%%   DAX_CONT_EXPORT
%%   void AllocateArrayForOutput(ContainerType &daxNotUsed(controlArray),
%%                               dax::Id numberOfValues) {
%%     // Assuming controlArray uses the same arrays as this.
%%     DAX_ASSERT_CONT(this->ArraysValid);

%%     // This implementation of allocate, which allocates the same amount in both arrays, is
%%     // arbitrary. It could, for example, leave the size of Array1 alone and change the size
%%     // of Array2. Or, probably most likely, it could simply throw an error and state that
%%     // this operation is invalid.
%%     dax::Id half = numberOfValues/2;
%%     this->ExecutionPortal
%%         = PortalExecution(this->Array1.PrepareForOutput(numberOfValues-half),
%%                           this->Array2.PrepareForOutput(half));
%%     this->ExecutionPortalValid = true;
%%     this->ExecutionPortalConstValid = false;
%%   }

%%   DAX_CONT_EXPORT
%%   void RetrieveOutputData(ContainerType &daxNotUsed(controlArray)) const {
%%     // Implementation of this method should be unnecessary. The internal
%%     // first and second array handles should automatically retrieve the
%%     // output data as necessary.
%%   }

%%   template<typename IteratorTypeControl>
%%   DAX_CONT_EXPORT
%%   void CopyInto(IteratorTypeControl dest) const {
%%     DAX_ASSERT_CONT(this->ArraysValid);
%%     this->Array1->CopyInto(dest);
%%     this->Array2->CopyInto(dest + this->Array1.GetNumberOfValues());
%%   }

%%   DAX_CONT_EXPORT
%%   void Shrink(dax::Id numberOfValues) {
%%     DAX_ASSERT_CONT(this->ArraysValid);
%%     if (numberOfValues < this->Array1.GetNumberOfValues())
%%       {
%%       this->Array1.Shrink(numberOfValues);
%%       this->Array2.Shrink(0);
%%       }
%%     else
%%       {
%%       this->Array2.Shrink(numberOfValues - this->Array1.GetNumberOfValues());
%%       }
%%   }

%%   DAX_CONT_EXPORT
%%   PortalExecution GetPortalExecution() {
%%     DAX_ASSERT_CONT(this->ExecutionPortalValid);
%%     return this->ExecutionPortal;
%%   }

%%   DAX_CONT_EXPORT
%%   PortalConstExecution GetPortalConstExecution() const {
%%     DAX_ASSERT_CONT(this->ExecutionPortalConstValid);
%%     return this->ExecutionPortalConst;
%%   }

%%   DAX_CONT_EXPORT
%%   void ReleaseResources() {
%%     DAX_ASSERT_CONT(this->ArraysValid);
%%     this->Array1.ReleaseResourcesExecution();
%%     this->Array2.ReleaseResourcesExecution();
%%     this->ExecutionPortalValid = false;
%%     this->ExecutionPortalConstValid = false;
%%   }

%% private:
%%   ArrayHandleType1 Array1;
%%   ArrayHandleType2 Array2;
%%   bool ArraysValid;
%%   PortalConstExecution ExecutionPortalConst;
%%   bool ExecutionPortalConstValid;
%%   PortalExecution ExecutionPortal;
%%   bool ExecutionPortalValid;
%% };

%% }
%% }
%% } // namespace dax::cont::internal
%% \end{daxexample}

%% \index{array~transfer|)}

%% The final step to make a derived container is to create a mechanism to
%% construct an \textidentifier{ArrayHandle} with a container derived from the
%% desired arrays. This can be done by creating a trivial subclass of
%% \daxcont{ArrayHandle} that simply constructs the array handle to the state
%% of an existing container. It uses a protected constructor of
%% \daxcont{ArrayHandle} that accepts a constructed container, array transfer,
%% and flags on the status of the control and execution arrays. \fix{There
%%   remains an issue with the concept template matching mechanism that can
%%   cause this subclass to fail with these ArrayHandle subclasses.}

%% \begin{daxexample}{\textidentifier{ArrayHandle} for derived container of concatenated arrays.}
%% template<typename ArrayHandleType1, typename ArrayHandleType2>
%% class ArrayHandleConcatenate
%%   : public dax::cont::ArrayHandle<
%%       typename ArrayHandleType1::ValueType,
%%       ArrayContainerControlTagConcatenate<ArrayHandleType1,ArrayHandleType2>,
%%       typename ArrayHandleType1::DeviceAdapterTag>
%% {
%%   typedef ArrayContainerControlTagConcatenate<ArrayHandleType1,ArrayHandleType2>
%%       ContainerTag;
%%   typedef dax::cont::ArrayHandle<
%%       typename ArrayHandleType1::ValueType,
%%       ContainerTag,
%%       typename ArrayHandleType1::DeviceAdapterTag> Superclass;
%%   typedef dax::cont::internal::ArrayContainerControl<T,ContainerTag> ContainerType;
%%   typedef dax::cont::internal::ArrayTransfer<
%%       typename ArrayHandleType1::ValueType,
%%       ContainerTag,
%%       typename ArrayHandleType1::DeviceAdapterTag> TransferType;

%% public:
%%   ArrayHandleConcatenate(const ArrayHandleType1 &array1, const ArrayHandleType2 &array2)
%%     : Superclass(ContainerType(array1, array2),
%%                  true,
%%                  TransferType(array1, array2),
%%                  false)
%%   {  }
%% };
%% \end{daxexample}

%% \index{container!derived|)}
%% \index{array~handle!derived|)}

\index{container|)}
\index{array~handle!container|)}

\index{array~handle|)}

%% \section{Grid Structures}
%% \label{sec:GridStructures}

%% The Dax toolkit provides containers for topologies. The topologies are
%% built on the previously described data structures (mostly array handles)
%% and are intentionally simplistic to simplify the adaptation to other
%% structures.

%% The grid structures are independent classes. They have no common
%% superclass. However, they do have some elements that are expected to be
%% common across all grids classes, which can be used in a templated
%% environment.

%% All grid structures have methods named \index{GetNumberOfPoints}
%% \textcode{GetNumberOfPoints} and \index{GetNumberOfCells}
%% \textcode{GetNumberOfCells}. These methods, of course, return the number of
%% points or cells in the grid structure.

%% \index{GetPointCoordiantes} All grid structures have a method called
%% \textcode{GetPointCoordinates}. This method returns an array handle that
%% contains the spatial coordinates for all the points in the mesh. Topologies
%% with implicit connections might return an array with an implicit or derived
%% container (meaning that the data is functionally defined rather than stored
%% in memory), but the arrays behave the same regardless. The type of the
%% array returned by \textcode{GetPointCoordinates} is specified by the type
%% \textcode{PointCoordinatesType} defined in the grid class. This will be
%% either a \textcode{typedef} of an \textidentifier{ArrayHandle} with
%% specific template parameters or a subclass of
%% \textidentifier{ArrayHandle}.

%% \index{ComputePointCoordinates} It is also possible to query the point
%% coordinates for any given point with the \textcode{ComputePointCoordinates}
%% method. This method is mainly provided for testing purposes. Most point
%% coordinate operations should be performed in the execution environment.

%% The \textcode{GetPointCoordinates} method is most useful for invoking an
%% operation on the point coordinates as a field on points. We have seen this
%% method used on the examples of the elevation worklet.

%% \begin{daxexample}{Processing point coordinates from an unknown grid type.}
%% template<typename GridType>
%% DAX_CONT_EXPORT
%% void Elevation(const GridType &grid,
%%                dax::cont::ArrayHandle<dax::Scalar> &outPointElevation)
%% {
%%   dax::cont::DispatcherMapField<dax::worklet::Elevation> dispatcher;
%%   dispatcher.Invoke(grid.GetPointCoordinates(), outPointElevation);
%% }
%% \end{daxexample}

%% Each grid structure contains a particular type of cell. Each grid structure
%% defines a type named \index{CellTag} \textcode{CellTag} that identifies the
%% type of cell stored. Cell types and operations that can be performed in the
%% execution environment are described in
%% Section~\ref{sec:CellsAndOperations}.

%% All grid structures also have the facilities to pack information to be sent
%% to the execution environment. There is a type defined in the grid class
%% called \textcode{TopologyStructConstExecution} for read-only input data and
%% a \textcode{PrepareForInput} method to build the structure. Likewise, there
%% is a \textcode{TopologyStructExecution} type and
%% \textcode{PrepareForOutput} method for output data.

%% The execution structures, however, differ significantly. Typically, these
%% facilities are handled internally within Dax to pass data to worklets.

%% \subsection{Uniform Grid}

%% \index{uniform~grid|(}

%% A uniform grid is stored in a \daxcont{UniformGrid} class. A uniform grid
%% is a topology structure where its points form a regular 3D array. The 3D
%% array of points are axis aligned, and the spacing is uniform along each
%% dimension. Adjacent are connected together in \index{voxel} \keyterm{voxel}
%% cells, which are simply axis aligned hexahedra.

%% The topology of a uniform grid is completely implicit and specified with
%% three pieces of information. First, the extent, stored in a \dax{Extent3}
%% structure, specifies the minimum and maximum indices of the array. Second,
%% the origin, stored in a \dax{Vector3}, gives the point coordinates of the
%% point at index $[0,0,0]$ (which may not actually be in the extent of the
%% grid). Third, the spacing, stored in a \dax{Vector3}, gives the amount of
%% space between adjacent points in each dimension.

%% The uniform grid class is templated on the device adapter for which it is
%% being used. Its prototype looks as follows.

%% \begin{daxexample}{Prototype for \protect\daxcont{UniformGrid}.}
%% template <class DeviceAdapterTag = DAX_DEFAULT_DEVICE_ADAPTER_TAG>
%% class UniformGrid;
%% \end{daxexample}

%% The \daxcont{UniformGrid} class provides the following features.
%% \begin{description}
%% \item[\textcode{CellTag}] A type that identifies what kind of cell is
%%   stored in this class. Always set to \dax{CellTagVoxel}.
%% \item[\textcode{GetExtent}] A method that returns a \dax{Extent3}
%%   specifying the extent of the 3 dimensional indices.
%% \item[\textcode{SetExtent}] A method that sets the extent of the 3
%%   dimensional indices. There are two versions of \textcode{SetExtent}: one
%%   that accepts a \dax{Extent3} object and another that accepts two
%%   \dax{Id3} objects specifying the minimum and maximum indices.
%% \item[\textcode{GetOrigin}] A method that returns a \dax{Vector3}
%%   specifying the coordinates for the origin of the grid.
%% \item[\textcode{SetOrigin}] A method that accepts a \dax{Vector3} as a
%%   parameter to set the coordinates for the origin of the grid.
%% \item[\textcode{GetSpacing}] A method that returns a \dax{Vector3}
%%   specifying the spacing between adjacent points along each dimension.
%% \item[\textcode{SetSpacing}] A method that accepts a \dax{Vector3} as a
%%   parameter to set the spacing between adjacent points along each
%%   dimension.
%% \item[\textcode{GetNumberOfPoints}] A method that returns the number of
%%   points in the grid.
%% \item[\textcode{GetNumberOfCells}] A method that returns the number of
%%   cells in the grid.
%% \item[\textcode{ComputePointIndex}] A convenience method that takes a
%%   \dax{Id3} representing the 3 dimensional coordinates of a point and
%%   returns the one dimensional index for the point. The 1 dimensional index
%%   corresponds to the index for point field arrays contained in
%%   \daxcont{ArrayHandle} objects.
%% \item[\textcode{ComputeCellIndex}] A convenience method that takes a
%%   \dax{Id3} representing the 3 dimensional coordinates of a cell and
%%   returns the one dimensional index for the cell. The 1 dimensional index
%%   corresponds to the index for cell field arrays contained in
%%   \daxcont{ArrayHandle} objects.
%% \item[\textcode{ComputePointLocation}] A convenience method that takes a 1
%%   dimensional point index and returns the corresponding 3 dimensional index
%%   as a \dax{Id3}. This method performs the inverse operation of
%%   \textcode{ComputePointIndex}.
%% \item[\textcode{ComputeCellLocation}] A convenience method that takes a 1
%%   dimensional cell index and returns the corresponding 3 dimensional index
%%   as a \dax{Id3}. This method performs the inverse operation of
%%   \textcode{ComputeCellIndex}.
%% \item[\textcode{ComputePointCoordinates}] A convenience method that
%%   returns the spatial coordinates for a given point. This method is
%%   overloaded to accept either a 1 dimensional index as a \dax{Id} or a 3
%%   dimensional index as a \dax{Id3}.
%% \item[\textcode{GetPointCoordinates}] Returns a \daxcont{ArrayHandle}
%%   containing spatial coordinates for each point. This array can be used as
%%   a field when invoking worklets. The array is implicit.
%% \item[\textcode{PointCoordinatesType}] The type returned by
%%   \textcode{GetPointCoordinates}. It is a specialization of
%%   \textidentifier{ArrayHandle}.
%% \item[\textcode{TopologyStructConstExecution}] A memory copyable structure
%%   holding the state of the uniform grid that can be used in the execution
%%   environment.
%% \item[\textcode{PrepareForInput}] A method that returns a
%%   \textcode{TopologyStructConstExecution} object to pass to the execution
%%   environment. This method is typically only used internally within the Dax
%%   toolkit.
%% \end{description}

%% \index{uniform~grid|)}

%% \subsection{Unstructured Grid}

%% \index{unstructured~grid|(}

%% An unstructured grid is stored in a \daxcont{UnstructuredGrid} class. An
%% unstructured grid is a topology with a collection of cells connected in
%% arbitrary ways. It first defines a list of points. It then has a connection
%% list that specifies for each cell the points that comprise the vertices for
%% each cell. The \daxcont{UnstructuredGrid} class is limited to containing
%% cells of only one type.

%% The topology of an unstructured grid is defined with a point coordinates
%% array and a cell connections array. The point coordinates array is an array
%% of \dax{Vertex3} values containing one for each point. The cell connections
%% array is an array of \dax{Id} values. The length of this array is the
%% number of cells times the number of vertices per cell. The connections for
%% a particular cell are grouped together in adjacent array values. The cell
%% connections are given in CGNS order\lcite{CGNS}. An example cell connection
%% array is given in Figure~\ref{fig:CellConnections}.

%% \begin{figure}[htb]
%%   \centering
%%   \includegraphics{images/CellConnections}
%%   \caption{The cell connection array for a simple triangle mesh.}
%%   \label{fig:CellConnections}
%% \end{figure}

%% The unstructured grid class is templated on the cell type
%% (\dax{CellTagHexahedron}, \dax{CellTagLine}, \dax{CellTagQuadrilateral},
%% \dax{CellTagTetrahedron}, \dax{CellTagTriangle}, \dax{CellTagVertex}, or
%% \dax{CellTagWedge}) the container for cell connections, the container for
%% the point coordinate array, and the device adapter. Its prototype looks as
%% follows.

%% \begin{daxexample}{Prototype for \protect\daxcont{UnstructuredGrid}.}
%% template <
%%     typename CellT,
%%     class CellConnectionsContainerControlTag = DAX_DEFAULT_ARRAY_CONTAINER_CONTROL_TAG,
%%     class PointsArrayContainerControlTag = DAX_DEFAULT_ARRAY_CONTAINER_CONTROL_TAG,
%%     class DeviceAdapterTag = DAX_DEFAULT_DEVICE_ADAPTER_TAG>
%% class UnstructuredGrid;
%% \end{daxexample}

%% The \daxcont{UnstructuredGrid} class provides the following features.
%% \begin{description}
%% \item[\textcode{CellTag}] A type that identifies what kind of cell is
%%   stored in this class. Always set to the first template parameter.
%% \item[\textcode{CellConnectionsType}] The type of the \daxcont{ArrayHandle}
%%   used to store cell connection indices.
%% \item[\textcode{PointCoordinatesType}] The type of the
%%   \daxcont{ArrayHandle} used to store point coordinates.
%% \item[\textcode{GetCellConnections}] A method used to get the array handle
%%   for the cell connections.
%% \item[\textcode{SetCellConnetions}] A method used to set the array handle
%%   for the cell connections.
%% \item[\textcode{GetPointCoordinates}] A method used to get the array handle
%%   for point coordinates.
%% \item[\textcode{SetPointCoordinates}] A method used to set the array handle
%%   for point coordinates.
%% \item[\textcode{ComputePointCoordinates}] A convenience method that takes a
%%   point index and returns the point coordinates at that index. The actual
%%   value is pulled from the point coordinates array.
%% \item[\textcode{GetNumberOfPoints}] A method that returns the number of
%%   points in the grid.
%% \item[\textcode{GetNumberOfCells}] A method that returns the number of
%%   cells in the grid.
%% \item[\textcode{TopologyStructExecution}] A memory copyable structure
%%   holding the state of the uniform grid that can be used in the execution
%%   environment.
%% \item[\textcode{TopologyStructConstExecution}] A read-only (const) form of
%%   \textcode{TopologyStructuExecution}.
%% \item[\textcode{PrepareForInput}] A method that returns a
%%   \textcode{TopologyStructConstExecution} object to pass to the execution
%%   environment. This method is typically only used internally within the Dax
%%   toolkit.
%% \item[\textcode{PrepareForOutput}] A method that returns a
%%   \textcode{TopologyStructExecution} object to pass to the execution
%%   environment. This method is typically only used internally within the Dax
%%   toolkit.
%% \end{description}

%% \index{unstructured~grid|)}

%% \section{Dispatchers}
%% \label{sec:Dispatchers}

%% Worklets, both those provided by the Dax toolkit as listed in
%% Section~\ref{sec:ProvidedWorklets} and ones created by a user as described
%% in Section~\ref{sec:CreatingWorklets}, are instantiated in the control
%% environment and run in the execution environment. This means that the
%% control environment must have a means to \index{invoke}\keyterm{invoke}
%% worklets that start running in the execution environment.

%% This invocation is done through a set of
%% \index{dispatcher}\keyterm{dispatcher} objects. A dispatcher object is an
%% object in the control environment that has an instance of a worklet and can
%% invoke that worklet with a set of arguments. There are multiple types of
%% dispatcher objects, each corresponding to a type of worklet object. All
%% dispatcher objects have at least two template parameters: the worklet class
%% being invoked, which is always the first argument, and the device adapter
%% tag, which is always the last argument and will be set to the default
%% device adapter if not specified.

%% All dispatcher classes have a method named \textcode{Invoke} that launches
%% the worklet in the execution environment.  The arguments to
%% \textcode{Invoke} must match those in the control signature of the worklet
%% held by the dispatcher.

%% The following is a list of the dispatchers defined in the Dax toolkit. The
%% dispatcher classes corresponded to list of worklet types as specified in
%% Section~\ref{sec:WorkletTypes} starting on
%% page~\pageref{sec:WorkletTypes}. See that section for more details and
%% examples of using these dispatcher classes.

%% \begin{description}
%% \item[\daxcont{DispatcherMapField}] The dispatcher used in conjunction with
%%   a worklet that subclasses \daxexec{WorkletMapField}. The class has two
%%   template arguments: the worklet type and the device adapter (optional).
%% \item[\daxcont{DispatcherMapCell}] The dispatcher used in conjunction with
%%   a worklet that subclasses \dax{WorkletMapCell}. The class has two
%%   template arguments: the worklet type and the device adapter (optional).
%% \item[\daxcont{DispatcherGenerateTopology}] The dispatcher used in
%%   conjunction with a worklet that subclasses
%%   \dax{WorkletGenerateTopology}. The class has three template arguments: the
%%   worklet type, the type of array handle containing the count of the number
%%   of cells being generated (optional), and the device adapter
%%   (optional). The default type of the count array handle is
%%   \daxcont{ArrayHandle}\textcode{<}\dax{Id}\textcode{>}. An instance of the
%%   count array handle must be provided in the constructor of
%%   \daxcont{DispatcherGenerateTopology}.
%% \item[\daxcont{DispatcherInterpolatedCell}] The dispatcher used in
%%   conjunction with a worklet that subclasses
%%   \dax{WorkletInterpolatedCell}. The class has three template arguments: the
%%   worklet type, the type of array handle containing the count of the number
%%   of cells being generated (optional), and the device adapter
%%   (optional). The default type of the count array handle is
%%   \daxcont{ArrayHandle}\textcode{<}\dax{Id}\textcode{>}. An instance of the
%%   count array handle must be provided in the constructor of
%%   \daxcont{DispatcherInterpolatedCell}.
%% \item[\daxcont{DispatcherGenerateKeysValues}] The dispatcher used in
%%   conjunction with a worklet that subclasses
%%   \dax{WorkletGenerateKeysValues}. The class has three template arguments:
%%   the worklet type, the type of array handle containing the count of the
%%   number of key-values being generated (optional), and the device adapter
%%   (optional). The default type of the count array handle is
%%   \daxcont{ArrayHandle}\textcode{<}\dax{Id}\textcode{>}. An instance of the
%%   count array handle must be provided in the constructor of
%%   \daxcont{DispatcherGenerateKeysValues}.
%% \item[\daxcont{DispatcherReduceKeysValues}] The dispatcher used in
%%   conjunction with a worklet that subclasses
%%   \dax{WorkletReduceKeysValues}. The class has three template arguments:
%%   the worklet type, the type of array handle containing the keys
%%   (optional), and the device adapter (optional). The default type of the
%%   key array handle is
%%   \daxcont{ArrayHandle}\textcode{<}\dax{Id}\textcode{>}. An instance of the
%%   key array handle must be provided in the constructor of
%%   \daxcont{DispatcherReduceKeysValues}.
%% \end{description}

%% \section{Timers}
%% \label{sec:Timers}

%% \index{timer|(}

%% It is often the case that you need to measure the time it takes for an
%% operation to happen. This could be for performing measurements for
%% algorithm study or it could be to dynamically adjust scheduling.

%% Performing timing in a multi-threaded environment can be tricky because
%% operations happen asynchronously. In the Dax control environment timing is
%% simplified because the control environment operates on a single
%% thread. However, operations invoked in the execution environment may run
%% asynchronously to operations in the control environment.

%% To ensure that accurate timings can be made, Dax provides a \daxcont{Timer}
%% class that is templated on the device adapter to provide an accurate
%% measurement of operations that happen on the device. The timing starts when
%% the \textidentifier{Timer} is constructed. The time elapsed can be
%% retrieved with a call to the \textcode{GetElapsedTime} method. This method
%% will block until all operations in the execution environment complete so as
%% to return an accurate time. The timer can be restarted with a call to the
%% \textcode{Reset} method.

%% \begin{daxexample}{Using \protect\daxcont{Timer}.}
%% dax::cont::UniformGrid<> grid;
%% grid.SetExtent(dax::make_Id3(0, 0, 0), dax::make_Id3(99, 99, 99));
%% grid.SetOrigin(dax::make_Vector3(0.0, 0.0, 0.0));
%% grid.SetSpacing(dax::make_Vector3(1.0, 1.0, 1.0));

%% dax::cont::ArrayHandle<dax::Scalar> results;
%% dax::cont::DispatchMapField<dax::worklet::Elevation> dispatcher;

%% dax::cont::Timer<> timer;
%% dispatcher.Invoke(grid.GetPointCoordinates(), results);
%% // This call makes sure data is pulled back to the host in a host/device architecture.
%% results.GetPortalConstControl();
%% dax::Scalar elapsedTime = timer.GetElapsedTime();

%% std::cout << "Time to run elevation: " << elapsedTime << std::endl;
%% \end{daxexample}

%% \index{timer|)}

%% \section{Error Handling}
%% \label{sec:ErrorHandlingControl}

%% \index{errors|(}

%% The Dax toolkit uses exceptions to report errors. All exceptions thrown by
%% Dax will be a subclass of \daxcont{Error}. For simple error reporting, it
%% is possible to simply catch a \daxcont{Error} and report the error message
%% string reported by the \textcode{GetMessage} method.

%% \begin{daxexample}{Simple error reporting.}
%% #include <dax/cont/Error.h>

%% int main(int argc, char **argv)
%% {
%%   try
%%     {
%%     // Do something cool with Dax
%%     // ...
%%     }
%%   catch (dax::cont::Error error)
%%     {
%%     std::cout << error.GetMessage() << std::endl;
%%     return 1;
%%     }
%%   return 0;
%% }
%% \end{daxexample}

%% There are two subclasses to \daxcont{Error}. These are
%% \daxcont{ErrorExecution} and \daxcont{ErrorControl}, and they represent
%% errors that happen in the respective execution and control environments.

%% Readers familiar with parallel programming will probably note the
%% difficulty in raising errors in multi-threaded execution like what happens
%% in the execution environment. In fact some devices, like CUDA devices, do
%% not support exceptions at all. Dax handles the error reporting in the
%% execution environment by flagging an error when it occurs and then throwing
%% an error in the control environment after all threads have terminated. This
%% means that the amount of execution that happens after an error is flagged
%% is indeterminate and any output values should be considered incorrect.

%% The \daxcont{ErrorControl} class is also broken down into several
%% subclasses that can be independently caught to handle different types of
%% errors. The following control errors exist and may be thrown.
%% \begin{description}
%% \item[\daxcont{ErrorControlAssert}] \index{assert} Thrown when an assertion
%%   fails, meaning a Dax operation reached an unexpected state. The header
%%   file \daxheader{dax/cont}{Assert.h} defines a macro named
%%   \daxmacro{DAX\_ASSERT\_CONT} that behaves much like the POSIX C assert
%%   macro except that a \textidentifier{ErrorControlAssert} is thrown rather
%%   than killing the application outright.
%% \item[\daxcont{ErrorControlBadValue}] Thrown when a Dax function or method
%%   encounters an invalid value that inhibits progress.
%% \item[\daxcont{ErrorControlInternal}] Thrown when Dax detects an internal
%%   state that should never be reached. This error usually indicates a bug in
%%   Dax or, at best, Dax failed to detect an invalid input it should have.
%% \item[\daxcont{ErrorControlOutOfMemory}] Thrown when a Dax function or
%%   method tries to allocate an array and fails.
%% \end{description}

%% \index{errors|)}

%% \index{device~adapter|(}

%% \section{Device Adapter Algorithms}
%% \label{sec:DeviceAdapterAlgorithms}

%% \index{device~adapter!algorithm|(}
%% \index{algorithm|(}

%% The Dax toolkit comes with the templated class
%% \daxcont{DeviceAdapterAlgorithm} that provides a set of algorithms that can
%% be invoked in the control environment and are run on the execution
%% environment. The template has a single argument that specifies the device
%% adapter tag.

%% \begin{daxexample}{Prototype for \protect\daxcont{DeviceAdapterAlgorithm}.}
%% namespace dax {
%% namespace cont {

%% template<class DeviceAdapterTag>
%% struct DeviceAdapterAlgorithm;

%% }
%% } // namespace dax::cont
%% \end{daxexample}

%% \textidentifier{DeviceAdapterAlgorithm} contains no state. It only has a
%% set of static methods that implement its algorithms. The following methods
%% are available.

%% \begin{description}
%% \item[\textcode{Copy}] \index{copy} Copies data from an input array to an
%%   output array. The copy takes place in the execution environment.
%% \item[\textcode{LowerBounds}] \index{lower~bounds} The
%%   \textcode{LowerBounds} method takes three arguments. The first argument
%%   is an \textidentifier{ArrayHandle} of sorted values. The second argument
%%   is another \textidentifier{ArrayHandle} of items to find in the first
%%   array. \textcode{LowerBounds} find the index of the first item that is
%%   greater than or equal to the target value, much like the
%%   \textcode{std::lower\_bound} STL algorithm. The results are returned in
%%   an \textidentifier{ArrayHandle} given in the third argument.

%%   There are two specializations of \textcode{LowerBounds}. The first takes
%%   an additional comparison function that defines the less-than
%%   operation. The second takes only two parameters. The first is an
%%   \textidentifier{ArrayHandle} of sorted \dax{Id}s and the second is an
%%   \textidentifier{ArrayHandle} of \dax{Id}s to find in the first list. The
%%   results are written back out to the second array. This second
%%   specialization is useful for inverting index maps.
%% \item[\textcode{ScanInclusive}] \index{scan!inclusive} The
%%   \textcode{ScanInclusive} method takes an input and an output
%%   \textidentifier{ArrayHandle} and performs a running sum on the input
%%   array. The first value in the output is the same as the first value in
%%   the input. The second value in the output is the sum of the first two
%%   values in the input. The third value in the output is the sum of the
%%   first three values of the input, and so on. \textcode{ScanInclusive}
%%   returns the sum of all values in the input.
%% \item[\textcode{ScanExclusive}] \index{scan!exclusive} The
%%   \textcode{ScanExclusive} method takes an input and an output
%%   \textidentifier{ArrayHandle} and performs a running sum on the input
%%   array. The first value in the output is always 0. The second value in the
%%   output is the same as the first value in the input. The third value in
%%   the output is the sum of the first two values in the input. The fourth
%%   value in the output is the sum of the first three values of the input,
%%   and so on. \textcode{ScanExclusive} returns the sum of all values in the
%%   input.
%% \item[\textcode{Schedule}] \index{schedule} The \textcode{Schedule} method
%%   takes a functor as its first argument and invokes it a number of times
%%   specified by the second argument. It should be assumed that each
%%   invocation of \textcode{Schedule} occurs on a separate thread although in
%%   practice there could be some thread sharing.

%%   There are two versions of the \textcode{Schedule} method. The first
%%   version takes a \dax{Id} and invokes the functor that number of
%%   times. The second version takes a \dax{Id3} and invokes the functor once
%%   for every entry in a 3D array of the given dimensions.

%%   The functor is expected to be an object with a const overloaded
%%   parentheses operator. The operator takes as a parameter the index of the
%%   invocation, which is either a \dax{Id} or a \dax{Id3} depending on what
%%   version of \textcode{Schedule} is being used. The functor must also
%%   provide a method named \textcode{SetErrorMessageBuffer} that accepts an
%%   argument of type \daxexecinternal{ErrorMessageBuffer}. If any errors
%%   occur during the invocations of the functor, it should call the
%%   \textcode{RaiseError} method of the
%%   \textidentifier{ErrorMessageBuffer}. That will cause the
%%   \textcode{Schedule} method to (eventually) throw a
%%   \daxcont{ErrorExecution} exception.
%% \item[\textcode{Sort}] \index{sort} The \textcode{Sort} method provides an
%%   unstable sort of an array. There are two forms of the \textcode{Sort}
%%   method. The first takes an \textidentifier{ArrayHandle} and sorts the
%%   values in place. The second takes an additional argument that is a
%%   functor that provides the comparison operation for the sort.
%% \item[\textcode{SortByKey}] \index{sort!by key} The \textcode{SortByKey}
%%   method works similarly to the \textcode{Sort} method except that it takes
%%   two \textidentifier{ArrayHandle}s: an array of keys and a corresponding
%%   array of values. The sort orders the array of keys in ascending values
%%   and also reorders the values so they remain paired with the same
%%   key. Like \textcode{Sort}, \textcode{SortByKey} has a version that sorts
%%   by the default less-than operator and a version that accepts a custom
%%   comparison functor.
%% \item[\textcode{StreamCompact}] \index{stream~compact} The
%%   \textcode{StreamCompact} method selectively removes values from an
%%   array. The first argument is an \textidentifier{ArrayHandle} to be
%%   compacted and the second argument is an \textidentifier{ArrayHandle} of
%%   equal size with flags indicating whether the corresponding input value is
%%   to be copied to the output. The third argument is an output
%%   \textidentifier{ArrayHandle} whose length is set to the number of true
%%   flags in the stencil and the passed values are put in order to the output
%%   array.

%%   There is also a second form of \textidentifier{StreamCompact} that only
%%   has the stencil and output as arguments. In this version, the output gets
%%   the corresponding index of where the input should be taken from.
%% \item[\textcode{Synchronize}] \index{synchronize} The
%%   \textidentifier{Synchronize} method waits for any asynchronous operations
%%   running on the device to complete and then returns.
%% \item[\textcode{Unique}] \index{unique} The \textcode{Unique} method
%%   removes all duplicate values in an \textidentifier{ArrayHandle}. The
%%   method will only find duplicates if they are adjacent to each other in
%%   the array. The easiest way to ensure that duplicate values are adjacent
%%   is to sort the array first.

%%   There are two versions of \textcode{Unique}. The first uses the equals
%%   operator to compare entries. The second accepts a binary functor to
%%   perform the comparisons.
%% \item[\textcode{UpperBounds}] \index{upper~bounds} The
%%   \textcode{UpperBounds} method takes three arguments. The first argument
%%   is an \textidentifier{ArrayHandle} of sorted values. The second argument
%%   is another \textidentifier{ArrayHandle} of items to find in the first
%%   array. \textcode{UpperBounds} find the index of the first item that is
%%   greater than to the target value, much like the
%%   \textcode{std::upper\_bound} STL algorithm. The results are returned in
%%   an \textidentifier{ArrayHandle} given in the third argument.

%%   There are two specializations of \textcode{UpperBounds}. The first takes
%%   an additional comparison function that defines the less-than
%%   operation. The second takes only two parameters. The first is an
%%   \textidentifier{ArrayHandle} of sorted \dax{Id}s and the second is an
%%   \textidentifier{ArrayHandle} of \dax{Id}s to find in the first list. The
%%   results are written back out to the second array. This second
%%   specialization is useful for inverting index maps.
%% \end{description}

%% \index{algorithm|)}
%% \index{device~adapter!algorithm|)}

%% \section{Implementing Device Adapters}
%% \label{sec:ImplementingDeviceAdapters}

%% The Dax toolkit comes with several implementations of device adapters so
%% that it may be ported to a variety of platforms. It is also possible to
%% provide new device adapters to support yet more devices, compilers, and
%% libraries. A new device adapter provides a tag, a class to manage arrays in
%% the execution environment, a collection of algorithms that run in the
%% execution environment, and (optionally) a timer.

%% Although not strictly necessary, the implementation of device adapters
%% within the Dax toolkit are divided into 3 header files with the names
%% \textfilename{DeviceAdapterTag\textasteriskcentered.h},
%% \textfilename{ArrayManagerExecution\textasteriskcentered.h} and
%% \textfilename{DeviceAdapterAlgorithm\textasteriskcentered.h}. The
%% \textfilename{DeviceAdapter\textasteriskcentered.h} that most code includes
%% is a trivial header that simply includes these other three files. For
%% example, the \daxheader{dax/tbb/cont}{DeviceAdapterTBB.h} for the Intel
%% Threading Building Blocks (TBB) device adapter simply contains the
%% following (with minutia like include guards removed).

%% \begin{daxexample}{Contents of \protect\daxheader{dax/tbb/cont}{DeviceAdapterTBB.h} file.}
%% #include <dax/tbb/cont/internal/DeviceAdapterTagTBB.h>
%% #include <dax/tbb/cont/internal/ArrayManagerExecutionTBB.h>
%% #include <dax/tbb/cont/internal/DeviceAdapterAlgorithmTBB.h>
%% \end{daxexample}

%% The reason the Dax toolkit breaks up the code for its device adapters this
%% way is that there is an interdependence between the implementation of each
%% device adapter and the mechanism to pick a default device adapter. Breaking
%% up the device adapter code in this way maintains an acyclic dependence among
%% header files.

%% \subsection{Tag}

%% The device adapter tag, as described in Section~\ref{sec:DeviceAdapterTag}
%% is a simple empty type that is used as a template parameter to identify the
%% device adapter. Every device adapter implementation provides one. The
%% device adapter tag is typically defined in an internal header file with a
%% prefix of \textfilename{DeviceAdapterTag}. Here is the implementation for
%% the TBB device adapter.

%% \begin{daxexample}{Implementation of the TBB device adapter tag.}
%% namespace dax {
%% namespace tbb {
%% namespace cont {

%% struct DeviceAdapterTagTBB {  };

%% }
%% }
%% } // namespace dax::tbb::cont
%% \end{daxexample}

%% \subsection{Array Manager Execution}

%% \index{device~adapter!array manager|(}
%% \index{array~manager~execution|(}
%% \index{execution~array~manager|(}

%% The Dax toolkit defines a template named
%% \daxcontinternal{ArrayManagerExecution} that is responsible for allocating
%% memory in the execution environment and copying data between the control
%% and execution environment. The execution array manager is typically defined
%% in an internal header file with a prefix of
%% \textfilename{ArrayManagerExecution}.

%% \begin{daxexample}{Prototype for \protect\daxcontinternal{ArrayManagerExecution}.}
%% namespace dax {
%% namespace cont {
%% namespace internal {

%% template<typename T, class ArrayContainerControlTag, class DeviceAdapterTag>
%% class ArrayManagerExecution;

%% }
%% }
%% } // namespace dax::cont::internal
%% \end{daxexample}

%% A device adapter must provide a partial specialization of
%% \textidentifier{ArrayManagerExecution} for its device adapter tag. The
%% implementation for \textidentifier{ArrayManagerExecution} is expected to
%% manage the resources for a single array, and it must provide the following
%% elements.

%% \begin{description}
%% \item[\textcode{ValueType}] A \textcode{typedef} of the type for each item
%%   in the array. This is the same type as the first template argument.
%% \item[\textcode{PortalType}] The type of an array portal that can be used
%%   in the execution environment to access the array.
%% \item[\textcode{PortalConstType}] A read-only (const) version of
%%   \textcode{PortalType}.
%% \item[\textcode{GetNumberOfValues}] A method that returns the number of
%%   values stored in the array. The results are undefined if the data has not
%%   been loaded or allocated.
%% \item[\textcode{LoadDataForInput}] A method that takes an array portal in
%%   the control environment, allocates a large enough array in the execution
%%   environment, and copies the data into that array. The data in the
%%   execution array is not expected to be changed. The allocated array can
%%   later be accessed via the \textcode{GetPortalConst} method.
%% \item[\textcode{LoadDataForInPlace}] A method that takes an array portal in
%%   the control environment, allocates a large enough array in the execution
%%   environment, and copies the data into that array. The data in the
%%   execution array is expected to be read and changed. The allocated array
%%   can later be accessed via the \textcode{GetPortal} and
%%   \textcode{GetPortalConst} methods.
%% \item[\textcode{AllocateArrayForOutput}] A method that takes an array
%%   container and a size and allocates an array in the execution environment
%%   of the specified size. The initial memory is uninitialized and can be
%%   accessed via the \textcode{GetPortal} method. The container argument can
%%   be used to allocate data when the control and execution share arrays, but
%%   this argument is often ignored.
%% \item[\textcode{RetrieveOutputData}] This method takes an array container,
%%   allocates memory in the control environment, and copies data from the
%%   execution environment into it.
%% \item[\textcode{CopyInto}] This method takes an STL-compatible iterator and
%%   copies data from the execution environment into it.
%% \item[\textcode{Shrink}] A method that adjusts the size of the array in the
%%   execution environment to something that is a smaller size. All the data
%%   up to the new length must remain valid. Typically, no memory is actually
%%   reallocated. Instead, a different end is marked.
%% \item[\textcode{GetPortal}] A method that returns an array portal
%%   that can be used in the execution environment. The portal was defined in
%%   either \textcode{LoadDataForInPlace} or
%%   \textcode{AllocateArrayForOutput}.
%% \item[\textcode{GetPortalConst}] A method that returns a read-only
%%   (const) array portal that can be used in the execution environment. The
%%   portal was defined in one of the load or allocate methods.
%% \item[\textcode{ReleaseResources}] A method that frees any resources
%%   (typically memory) in the execution environment.
%% \end{description}

%% Specializations of this template typically take on one of two forms. If the
%% control and execution environments have separate memory spaces, then this
%% class behaves by copying memory in methods such as
%% \textcode{PrepareForInput} and \textcode{RetrieveOutputData}. This might
%% require creating buffers in the control environment to efficiently move
%% data from control array portals.

%% However, if the control and execution environments share the same memory
%% space, the execution array manager can, and should, delegate all of its
%% operations to the \textidentifier{ArrayContainerControl} it is used
%% with. The Dax toolkit comes with a class called
%% \daxcontinternal{ArrayManagerExecutionShareWithControl} that provides the
%% implementation for an execution array manager that shares a memory space
%% with the control environment. In this case, making the
%% \textidentifier{ArrayManagerExecution} specialization be a trivial subclass
%% is sufficient. For example, here is the implementation of
%% \textidentifier{ArrayManagerExecution} for TBB.

%% \begin{daxexample}{Specialization of \textidentifier{ArrayManagerExecution} for TBB.}
%% #include <dax/tbb/cont/internal/DeviceAdapterTagTBB.h>

%% #include <dax/cont/internal/ArrayManagerExecution.h>
%% #include <dax/cont/internal/ArrayManagerExecutionShareWithControl.h>

%% namespace dax {
%% namespace cont {
%% namespace internal {

%% template <typename T, class ArrayContainerTag>
%% class ArrayManagerExecution
%%     <T, ArrayContainerTag, dax::tbb::cont::DeviceAdapterTagTBB>
%%     : public dax::cont::internal::ArrayManagerExecutionShareWithControl
%%         <T, ArrayContainerTag>
%% {
%% };

%% }
%% }
%% } // namespace dax::cont::internal
%% \end{daxexample}

%% \index{execution~array~manager|)}
%% \index{array~manager~execution|)}
%% \index{device~adapter!array manager|)}

%% \subsection{Algorithms}

%% \index{device~adapter!algorithm|(}
%% \index{algorithm|(}

%% A device adapter implementation must also provide a specialization of
%% \daxcont{DeviceAdapterAlgorithm}, which is documented in
%% Section~\ref{sec:DeviceAdapterAlgorithms}. The implementation for the
%% device adapter algorithms is typically placed in a header file with a
%% prefix of \textfilename{DeviceAdapterAlgorithm}.

%% Although there are many methods in
%% \textidentifier{DeviceAdapterAlgorithms}, it is seldom necessary to
%% implement them all. Instead, the Dax toolkit comes with
%% \daxcontinternal{DeviceAdapterAlgorithmGeneral} that provides generic
%% implementation for most of the required algorithms. By deriving the
%% specialization of \textidentifier{DeviceAdapterAlgorithm} from
%% \textidentifier{DeviceAdapterAlgorithmGeneral}, only the implementations
%% for \textcode{Schedule} and \textcode{Synchronize} need to be
%% implemented. All other algorithms can be derived from those.

%% That said, not all of the algorithms implemented in
%% \textidentifier{DeviceAdapterAlgorithmGeneral} are optimized for all types
%% of devices. Thus, it is worthwhile to provide algorithms optimized for the
%% specific device when possible. In particular, it is best to provide
%% specializations for the sort and scan algorithms.

%% The following example is a minimal implementation of the TBB device adapter
%% algorithms. The actual version that comes with the Dax toolkit contains
%% more enhancements.

%% \begin{daxexample}{Abbreviated implementation of \textidentifier{DeviceAdapterAlgorithm} for TBB.}
%% #include <dax/tbb/cont/internal/DeviceAdapterTagTBB.h>
%% #include <dax/tbb/cont/internal/ArrayManagerExecutionTBB.h>

%% #include <dax/cont/internal/DeviceAdapterAlgorithmGeneral.h>
%% #include <dax/exec/internal/IJKIndex.h>

%% #include <tbb/blocked_range.h>
%% #include <tbb/blocked_range3d.h>
%% #include <tbb/parallel_for.h>

%% namespace dax {
%% namespace cont {

%% template<>
%% struct DeviceAdapterAlgorithm<dax::tbb::cont::DeviceAdapterTagTBB> :
%%     dax::cont::internal::DeviceAdapterAlgorithmGeneral<
%%         DeviceAdapterAlgorithm<dax::tbb::cont::DeviceAdapterTagTBB>,
%%         dax::tbb::cont::DeviceAdapterTagTBB>
%% {
%% private:
%%   static const dax::Id TBB_GRAIN_SIZE = 128;

%%   template<class FunctorType>
%%   class ScheduleKernel
%%   {
%%   public:
%%     DAX_CONT_EXPORT ScheduleKernel(const FunctorType &functor)
%%       : Functor(functor)
%%     {  }

%%     DAX_CONT_EXPORT void SetErrorMessageBuffer(
%%         const dax::exec::internal::ErrorMessageBuffer &errorMessage)
%%     {
%%       this->ErrorMessage = errorMessage;
%%       this->Functor.SetErrorMessageBuffer(errorMessage);
%%     }

%%     DAX_EXEC_EXPORT
%%     void operator()(const ::tbb::blocked_range<dax::Id> &range) const {
%%       // The TBB device adapter causes array classes to be shared between
%%       // control and execution environment. This means that it is possible for
%%       // an exception to be thrown even though this is typically not allowed.
%%       // Throwing an exception from here is bad because there are several
%%       // simultaneous threads running. Get around the problem by catching the
%%       // error and setting the message buffer as expected.
%%       try
%%         {
%%         for (dax::Id index = range.begin(); index < range.end(); index++)
%%           {
%%           this->Functor(index);
%%           }
%%         }
%%       catch (dax::cont::Error error)
%%         {
%%         this->ErrorMessage.RaiseError(error.GetMessage().c_str());
%%         }
%%       catch (...)
%%         {
%%         this->ErrorMessage.RaiseError(
%%             "Unexpected error in execution environment.");
%%         }
%%     }
%%   private:
%%     FunctorType Functor;
%%     dax::exec::internal::ErrorMessageBuffer ErrorMessage;
%%   };

%% public:
%%   template<class FunctorType>
%%   DAX_CONT_EXPORT
%%   static void Schedule(FunctorType functor, dax::Id numInstances)
%%   {
%%     const dax::Id MESSAGE_SIZE = 1024;
%%     char errorString[MESSAGE_SIZE];
%%     errorString[0] = '\0';
%%     dax::exec::internal::ErrorMessageBuffer
%%         errorMessage(errorString, MESSAGE_SIZE);

%%     ScheduleKernel<FunctorType> kernel(functor);
%%     kernel.SetErrorMessageBuffer(errorMessage);

%%     ::tbb::blocked_range<dax::Id> range(0, numInstances, TBB_GRAIN_SIZE);

%%     ::tbb::parallel_for(range, kernel);

%%     if (errorMessage.IsErrorRaised())
%%       {
%%       throw dax::cont::ErrorExecution(errorString);
%%       }
%%   }

%% private:
%%   template<class FunctorType>
%%   class ScheduleKernelId3
%%   {
%%   public:
%%     DAX_CONT_EXPORT ScheduleKernelId3(const FunctorType &functor,
%%                                       const dax::Id3& dims)
%%       : Functor(functor),
%%         Dims(dims)
%%       {  }

%%     DAX_CONT_EXPORT void SetErrorMessageBuffer(
%%         const dax::exec::internal::ErrorMessageBuffer &errorMessage)
%%     {
%%       this->ErrorMessage = errorMessage;
%%       this->Functor.SetErrorMessageBuffer(errorMessage);
%%     }

%%     DAX_EXEC_EXPORT
%%     void operator()(const ::tbb::blocked_range3d<dax::Id> &range) const {
%%       try
%%         {
%%         dax::exec::internal::IJKIndex index(this->Dims);
%%         for( dax::Id k=range.pages().begin(); k!=range.pages().end(); ++k)
%%           {
%%           index.SetK(k);
%%           for( dax::Id j=range.rows().begin(); j!=range.rows().end(); ++j)
%%             {
%%             index.SetJ(j);
%%             for( dax::Id i=range.cols().begin(); i!=range.cols().end(); ++i)
%%               {
%%               index.SetI(i);
%%               this->Functor(index);
%%               }
%%             }
%%           }
%%         }
%%       catch (dax::cont::Error error)
%%         {
%%         this->ErrorMessage.RaiseError(error.GetMessage().c_str());
%%         }
%%       catch (...)
%%         {
%%         this->ErrorMessage.RaiseError(
%%             "Unexpected error in execution environment.");
%%         }
%%     }
%%   private:
%%     FunctorType Functor;
%%     dax::Id3 Dims;
%%     dax::exec::internal::ErrorMessageBuffer ErrorMessage;
%%   };

%% public:
%%   template<class FunctorType>
%%   DAX_CONT_EXPORT
%%   static void Schedule(FunctorType functor,
%%                        dax::Id3 rangeMax)
%%   {
%%     //we need to extract from the functor that uniform grid information
%%     const dax::Id MESSAGE_SIZE = 1024;
%%     char errorString[MESSAGE_SIZE];
%%     errorString[0] = '\0';
%%     dax::exec::internal::ErrorMessageBuffer
%%         errorMessage(errorString, MESSAGE_SIZE);

%%     //memory is generally setup in a way that iterating the first range
%%     //in the tightest loop has the best cache coherence.
%%     ::tbb::blocked_range3d<dax::Id> range(0, rangeMax[2],
%%                                           0, rangeMax[1],
%%                                           0, rangeMax[0]);

%%     ScheduleKernelId3<FunctorType> kernel(functor,rangeMax);
%%     kernel.SetErrorMessageBuffer(errorMessage);

%%     ::tbb::parallel_for(range, kernel);

%%     if (errorMessage.IsErrorRaised())
%%       {
%%       throw dax::cont::ErrorExecution(errorString);
%%       }
%%   }

%%   DAX_CONT_EXPORT static void Synchronize()
%%   {
%%     // Nothing to do. This device schedules all of its operations using a
%%     // split/join paradigm. This means that the if the control thread is
%%     // calling this method, then nothing should be running in the execution
%%     // environment.
%%   }

%% };

%% }
%% } // namespace dax::cont
%% \end{daxexample}

%% \index{algorithm|)}
%% \index{device~adapter!algorithm|)}

%% \subsection{Timer Implementation}

%% The Dax timer, described in Section~\ref{sec:Timers}, delegates to an
%% internal class named \daxcont{DeviceAdapterTimerImplementation}. The
%% interface for this class is the same as that for \daxcont{Timer}. A default
%% implementation of this templated class uses the system timer and the
%% \textcode{Synchronize} method in the device adapter algorithms.

%% However, some devices might provide alternate or better methods for
%% implementing timers. For example, the TBB library comes with a high
%% resolution timer that has better accuracy than the standard system
%% timers. Thus, the device adapter can optionally provide a specialization of
%% \textidentifier{DeviceAdapterTimerImplementation}, which is typically
%% placed in the same header file as the device adapter algorithms.

%% The following example is the implementation of the TBB timer
%% implementation.

%% \begin{daxexample}{Implementation of \textidentifier{DeviceAdapterTimerImplementation} for TBB.}
%% #include <dax/cont/DeviceAdapter.h>
%% #include <dax/tbb/cont/internal/DeviceAdapterTagTBB.h>

%% #include <tbb/tick_count.h>

%% namespace dax {
%% namespace cont {

%% template<>
%% class DeviceAdapterTimerImplementation<dax::tbb::cont::DeviceAdapterTagTBB>
%% {
%% public:
%%   DAX_CONT_EXPORT DeviceAdapterTimerImplementation()
%%   {
%%     this->Reset();
%%   }
%%   DAX_CONT_EXPORT void Reset()
%%   {
%%     dax::cont::DeviceAdapterAlgorithm<dax::tbb::cont::DeviceAdapterTagTBB>::Synchronize();
%%     this->StartTime = ::tbb::tick_count::now();
%%   }
%%   DAX_CONT_EXPORT dax::Scalar GetElapsedTime()
%%   {
%%     dax::cont::DeviceAdapterAlgorithm<dax::tbb::cont::DeviceAdapterTagTBB>::Synchronize();
%%     ::tbb::tick_count currentTime = ::tbb::tick_count::now();
%%     ::tbb::tick_count::interval_t elapsedTime = currentTime - this->StartTime;
%%     return static_cast<dax::Scalar>(elapsedTime.seconds());
%%   }

%% private:
%%   ::tbb::tick_count StartTime;
%% };

%% }
%% } // namespace dax::cont
%% \end{daxexample}

%% A word of warning about implementing timers. Although
%% \textcode{GetElapsedTime} returns a \dax{Scalar}, it is advisable to store
%% the internal timing in its native data format until the elapsed time is
%% recorded. This is because the times may be biased by a large value, and the
%% floating point number might not hold enough precision to get a precise
%% measurement between the start and end of the timer.

%% \subsection{Testing}

%% The implementation of a device adapter contains many components. To ensure
%% that all of its device adapters are working properly, the Dax toolkit
%% contains a complete test of all the components in
%% \daxheader{dax/cont/testing}{TestingDeviceAdapter.h}. Here is the
%% implementation for the TBB device adapter test, which plugs into the CMake
%% testing framework.

%% \begin{daxexample}{Test code for the TBB device adapter.}
%% #include <dax/tbb/cont/DeviceAdapterTBB.h>

%% #include <dax/cont/testing/TestingDeviceAdapter.h>

%% int UnitTestDeviceAdapterTBB(int, char *[])
%% {
%%   return dax::cont::testing::TestingDeviceAdapter
%%       <dax::tbb::cont::DeviceAdapterTagTBB>::Run();
%% }
%% \end{daxexample}

%% \index{device~adapter|)}

\index{control~environment|)}

