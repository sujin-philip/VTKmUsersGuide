% -*- latex -*-

\chapter{Control Environment}
\label{chap:ControlEnvironment}

\index{control~environment|(}
\index{environment!control|(}

The control environment is where code interfaces with applications and I/O
devices. The associated API is designed for users that want to use VTK-m to
analyze their data using provided or supplied worklets. Code for the
control environment is designed to run on a single thread (or one single
thread per process in an MPI job).

Most users of VTK-m will have some interaction with the control
environment, for you cannot define data structures or execute any
algorithms without it.

\section{Timers}
\label{sec:Timers}

\index{timer|(}

It is often the case that you need to measure the time it takes for an
operation to happen. This could be for performing measurements for
algorithm study or it could be to dynamically adjust scheduling.

Performing timing in a multi-threaded environment can be tricky because
operations happen asynchronously. In the VTK-m control environment timing
is simplified because the control environment operates on a single
thread. However, operations invoked in the execution environment may run
asynchronously to operations in the control environment.

To ensure that accurate timings can be made, VTK-m provides a
\vtkmcont{Timer} class that is templated on the device adapter to provide
an accurate measurement of operations that happen on the device. If not
template parameter is provided, the default device adapter is used.

The timing starts when the \textidentifier{Timer} is constructed. The time
elapsed can be retrieved with a call to the \textcode{GetElapsedTime}
method. This method will block until all operations in the execution
environment complete so as to return an accurate time. The timer can be
restarted with a call to the \textcode{Reset} method.

\fix{This example needs to be updated when something interesting can be
  invoked.}

\vtkmlisting{Using \protect\vtkmcont{Timer}.}{Timer.cxx}

\index{timer|)}

\section{Error Handling}
\label{sec:ErrorHandlingControl}

\index{errors|(}

VTK-m uses exceptions to report errors. All exceptions thrown by VTK-m will
be a subclass of \vtkmcont{Error}. For simple error reporting, it is
possible to simply catch a \vtkmcont{Error} and report the error message
string reported by the \textcode{GetMessage} method.

\vtkmlisting{Simple error reporting.}{CatchingErrors.cxx}

There are several subclasses to \vtkmcont{Error}. The specific subclass
gives an indication of the type of error that occured when the exception
was thrown. Catching one of these subclasses may help a program better
recover from errors.
\begin{description}
\item[\vtkmcont{ErrorControlAssert}] \index{assert} \index{errors!assert}
  Thrown when an assertion fails, meaning a VTK-m operation reached an
  unexpected state. The header file \vtkmheader{vtkm/cont}{Assert.h}
  defines a macro named \vtkmmacro{VTKM\_ASSERT\_CONT} that behaves much
  like the POSIX C assert macro except that a
  \textidentifier{ErrorControlAssert} is thrown rather than killing the
  application outright.
\item[\vtkmcont{ErrorControlBadAllocation}] Thrown when there is a problem
  accessing or manipulating memory. Often this is thrown when an allocation
  fails because there is insufficient memory, but other memory access
  errors can cause this to be thrown as well.
\item[\vtkmcont{ErrorControlBadType}] Thrown when VTK-m attempts to perform
  an operation on an object that is of an incompatible type.
\item[\vtkmcont{ErrorControlBadValue}] Thrown when a VTK-m function or
  method encounters an invalid value that inhibits progress.
\item[\vtkmcont{ErrorExecution}] \index{errors!execution~environment} Throw
  when an error is signaled in the execution environment for example when a
  worklet is being executed.
\item[\vtkmcont{ErrorControlInternal}] Thrown when VTK-m detects an
  internal state that should never be reached. This error usually indicates
  a bug in VTK-m or, at best, VTK-m failed to detect an invalid input it
  should have.
\item[\vtkmio{ErrorIO}] Thrown by a reader or writer when a file error is
  encountered.
\end{description}

\index{errors|)}

\index{environment!control|)}
\index{control~environment|)}

