% -*- latex -*-

\chapter{Creating Filters}
\label{chap:CreatingFilters}

\index{filter|(}

In Chapter~\ref{chap:Worklets} we discuss how to implement an algorithm in the VTK-m framework by creating a worklet.
Worklets might be straightforward to implement and invoke for those well familiar with the appropriate VTK-m API.
However, novice users have difficulty using worklets directly.
For simplicity, worklet algorithms are generally wrapped in what are called filter objects for general usage.
Chapter~\ref{chap:ProvidedFilters} introduces the concept of filters and documents those that come with the VTK-m library.
In this chapter we describe how to build new filter objects using the worklet examples introduced in Chapter~\ref{chap:Worklets}.

Unsurprisingly, the base filter objects are contained in the \vtkmfilter{} package.
The basic implementation of a filter involves subclassing one of the base filter objects and implementing the \textcode{DoExecute} method.
The \textcode{DoExecute} method performs the operation of the filter and returns the appropriate result object.

As with worklets, there are several flavors of filter types to address different operating behaviors although their is not a one-to-one relationship between worklet and filter types.
This chapter is sectioned by the different filter types with an example of implementations for each.

\section{Field Filters}

\index{filter!field|(}
\index{field filter|(}

As described in Section~\ref{sec:ProvidedFieldFilters} (starting on page~\pageref{sec:ProvidedFieldFilters}), field filters are a category of filters that generate a new fields.
These new fields are typically derived from one or more existing fields or point coordinates on the data set.
For example, mass, volume, and density are interrelated, and any one can be derived from the other two.

Field filters are implemented in classes that derive the \vtkmfilter{FilterField} base class.
\textidentifier{FilterField} is a templated class that has a single template argument, which is the type of the concrete subclass.

\begin{didyouknow}
  The convention of having a subclass be templated on the derived class' type is known as the Curiously Recurring Template Pattern (CRTP).
  In the case of \textidentifier{FilterField} and other filter base classes, \VTKm uses this CRTP behavior to allow the general implementation of these algorithms to run \textcode{DoExecute} in the subclass, which as we see in a moment is itself templated.
\end{didyouknow}

All \textidentifier{FilterField} subclasses must implement a \textcode{DoExecute} method.
The \textidentifier{FilterField} base class implements an \textcode{Execute} method (actually several overloaded versions of \textcode{Execute}), processes the arguments, and then calls the \textcode{DoExecute} method of its subclass.
The \textcode{DoExecute} method has the following 5 arguments.
\begin{itemize}
\item
  An input data set contained in a \vtkmcont{DataSet} object.
  (See Chapter~\ref{chap:DataSet} for details on \textidentifier{DataSet} objects.)
\item
  The field from the \textidentifier{DataSet} specified in the \textcode{Execute} method to operate on.
  The field is always passed as an instance of \vtkmcont{ArrayHandle}.
  (See Chapter~\ref{chap:ArrayHandle} for details on \textidentifier{ArrayHandle} objects.)
  The type of the \textidentifier{ArrayHandle} is generally not known until the class is used and requires a template type.
\item
  A \vtkmfilter{FieldMetadata} object that contains the associated metadata of the field not contained in the \textidentifier{ArrayHandle} of the second argument.
  The \textidentifier{FieldMetadata} contains information like the name of the field and what topological element the field is associated with (such as points or cells).
\item
  A policy class.
  (See Chapter~\ref{chap:Policies} for details on policy classes.)
  The type of the policy is generally to known until the class is used and requires a template type.
\item
  A device adapter tag.
  (See Chapter~\ref{chap:DeviceAdapter} for details on device adapters and their tags.)
  The type of the device adapter is generally to known until the class is used and requires a template type.
\end{itemize}

In this section we provide an example implementation of a field filter that wraps the ``magnitude'' worklet provided in Example~\ref{ex:UseWorkletMapField} (listed on page~\pageref{ex:UseWorkletMapField}).
By convention, filter implementations are split into two files.
The first file is a standard header file with a \textfilename{.h} extension that contains the declaration of the filter class without the implementation.
So we would expect the following code to be in a file named \textfilename{FieldMagnitude.h}.

\vtkmlisting[ex:MagnitudeFilterDeclaration]{Header declaration for a field filter.}{UseFilterField.cxx}

\index{filter!traits|(}
\index{traits!filter|(}

Notice that in addition to declaring the class for \textcode{FieldMagnitude}, Example~\ref{ex:MagnitudeFilterDeclaration} also specializes the \vtkmfilter{FilterTraits} templated class for \textcode{FieldMagnitude}.
The \textidentifier{FilterTraits} class, declared in \vtkmheader{vtkm/filter}{FilterTraits.h}, provides hints used internally in the base filter classes to modify its behavior based on the subclass.
A \textidentifier{FilterTraits} class is expected to define the following types.
\begin{description}
\item[\textcode{InputFieldTypeList}]
  A type list containing all the types that are valid for the input field.
  For example, a filter operating on positions in space might limit the types to three dimensional vectors.
  Type lists are discussed in detail in Section~\ref{sec:TypeLists}.
\end{description}

In the particular case of our \textcode{FieldMagnitude} filter, the filter expects to operate on some type of vector field.
Thus, the \textcode{InputFieldTypeList} is modified to a list of all standard floating point \textidentifier{Vec}s.

Once the filter class and (optionally) the \textidentifier{FilterTraits} are declared in the \textfilename{.h} file, the implementation filter is by convention given in a separate \textfilename{.hxx} file.
So the continuation of our example that follows would be expected in a file named \textfilename{FieldMagnitude.hxx}.
The \textfilename{.h} file near its bottom needs an include line to the \textfilename{.hxx} file.
This convention is set up because a near future version of \VTKm will allow the building of filter libraries containing default policies that can be used by only including the header declaration.

The implementation of \textcode{DoExecute} is straightforward.
A worklet is invoked to compute a new field array.
\textcode{DoExecute} then returns a newly constructed \vtkmfilter{ResultField} object.
The constructor to \textidentifier{ResultField} takes the following 5 arguments.
\begin{itemize}
\item
  The input data set.
  This is the same data set passed to the first argument of \textcode{DoExecute}.
\item
  The array containing the data for the new field, which was presumably computed by the filter.
\item
  The name of the new field.
\item
  The topological association (e.g. points or cells) of the new field.
  In the case where the filter is a simple operation on a field array, the association can usually be copied from the \textidentifier{FieldMetadata} passed to \textcode{DoExecute}.
\item
  The name of the element set the new field is associated with.
  This only has meaning if the new field is associated with cells.
  This usually can be copied from the \textidentifier{FieldMetadata} passed to \textcode{DoExecute}.
\end{itemize}

Note that all fields need a unique name, which is the reason for the third argument to the \textidentifier{ResultField} constructor.
The \vtkmfilter{FilterField} base class contains a pair of methods named \textcode{SetOutputFieldName} and \textcode{GetOutputFieldName} to allow users to specify the name of output fields.
The \textcode{DoExecute} method should respect the given output field name.
However, it is also good practice for the filter to have a default name if none is given.
This might be simply specifying a name in the constructor, but it is worthwhile for many filters to derive a name based on the name of the input field.

\vtkmlisting{Implementation of a field filter.}{FilterFieldImpl.cxx}

\index{traits!filter|)}
\index{filter!traits|)}

\index{field filter|)}
\index{filter!field|)}

\index{filter|)}
