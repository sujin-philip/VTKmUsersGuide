% -*- latex -*-

\chapter{Creating Filters}
\label{chap:CreatingFilters}

In Chapter~\ref{chap:Worklets} we discuss how to implement an algorithm in the VTK-m framework by creating a worklet.
Worklets might be straightforward to implement and invoke for those well familiar with the appropriate VTK-m API.
However, novice users have difficulty using worklets directly.
For simplicity, worklet algorithms are generally wrapped in what are called filter objects for general usage.
Chapter~\ref{chap:ProvidedFilters} introduces the concept of filters and documents those that come with the VTK-m library.
In this chapter we describe how to build new filter objects using the worklet examples introduced in Chapter~\ref{chap:Worklets}.

Unsurprisingly, the base filter objects are contained in the \vtkmfilter{} package.
The basic implementation of a filter involves subclassing one of the base filter objects and implementing the \textcode{DoExecute} method.
The \textcode{DoExecute} method performs the operation of the filter and returns the appropriate result object.

As with worklets, there are several flavors of filter types to address different operating behaviors although their is not a one-to-one relationship between worklet and filter types.
This chapter is sectioned by the different filter types with an example of implementations for each.
