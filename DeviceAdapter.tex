% -*- latex -*-

\chapter{Device Adapters}

\index{device~adapter|(}

As multiple vendors vie to provide accelerator-type processors, a great
variance in the computer architecture exists. Likewise, there exist
multiple compiler environments and libraries for these devices such as
CUDA, OpenMP, and various threading libraries. These compiler technologies
also vary from system to system.

To make porting among these systems at all feasible, we require a base
language support, and the language we use is C++. The majority of the code
in VTK-m is constrained to the standard C++ language constructs to minimize
the specialization from one system to the next.

Each device and device technology requires some level of code
specialization, and that specialization is encapsulated in a unit called a
\keyterm{device adapter}. Thus, porting VTK-m to a new architecture can be
done by adding only a device adapter.

The device adapter is shown diagrammatically as the connection between the
control and execution environments in Figure~\ref{fig:VTKmDiagram} on
page~\pageref{fig:VTKmDiagram}. The functionality of the device adapter
comprises two main parts: a collection of parallel algorithms run in the
execution environment and a module to transfer data between the control and
execution environments.

This chapter describes how tags are used to specify which devices to use
for operations within VTK-m. The chapter also outlines the features provided
by a device adapter that allow you to directly control a device. Finally,
we document how to create a new device adapter to port or specialize VTK-m
for a different device or system.


\section{Device Adapter Tag}
\label{sec:DeviceAdapterTag}

\index{device~adapter~tag|(}
\index{tag!device~adapter|(}

A device adapter is identified by a \keyterm{device adapter tag}. This tag,
which is simply an empty struct type, is used as the template parameter for
several classes in the VTK-m control environment and causes these classes
to direct their work to a particular device.

There are two ways to select a device adapter. The first is to make a
global selection of a default device adapter. The second is to specify a
specific device adapter as a template parameter.

\subsection{Default Device Adapter}
\label{sec:DefaultDeviceAdapter}

A default device adapter tag is specified in
\vtkmheader{vtkm/cont}{DeviceAdapter.h} (although it can also by specified
in many other VTK-m headers via header dependencies). If no other
information is given, VTK-m attempts to choose a default device adapter
that is a best fit for the system it is compiled on. VTK-m currently select
the default device adapter with the following sequence of conditions.

\begin{itemize}
\item \index{CUDA} If the source code is being compiled by CUDA, the CUDA
  device is used.
\item \index{OpenMP} If the CUDA compiler is not being used and the current
  compiler supports OpenMP, then the OpenMP device is used.
  \fix{Technically, OpenMP is not yet supported in VTK-m, so this will
    never actually be picked. But once it is implemented, this will be the
    chain.}
\item \index{Intel Threading Building Blocks} \index{TBB} If the compiler
  supports neither CUDA nor OpenMP and VTK-m was configured to use Intel
  Threading Building Blocks, then that device is used.
\item \index{serial} If no parallel device adapters are found, then VTK-m
  falls back to a serial device.
\end{itemize}

You can also set the default device adapter specifically by setting the
\vtkmmacro{VTKM\_DEVICE\_ADAPTER} macro. This macro must be set
\emph{before} including any VTK-m files. You can set
\vtkmmacro{VTKM\_DEVICE\_ADAPTER} to any one of the following.

\begin{description}
\item[\vtkmmacro{VTKM\_DEVICE\_ADAPTER\_SERIAL}] Performs all computation on
  the same single thread as the control environment. This device is useful
  for debugging. This device is always available.
\item[\vtkmmacro{VTKM\_DEVICE\_ADAPTER\_CUDA}] Uses a CUDA capable GPU
  device. For this device to work, VTK-m must be configured to use CUDA and
  the code must be compiled by the CUDA \textfilename{nvcc} compiler.
\item[\vtkmmacro{VTKM\_DEVICE\_ADAPTER\_OPENMP}] Uses OpenMP compiler
  extensions to run algorithms on multiple threads. For this device to
  work, VTK-m must be configured to use OpenMP and the code must be
  compiled with a compiler that supports OpenMP pragmas. \fix{Not currently
    implemented.}
\item[\vtkmmacro{VTKM\_DEVICE\_ADAPTER\_TBB}] Uses the Intel Threading
  Building Blocks library to run algorithms on multiple threads. For this
  device to work, VTK-m must be configured to use TBB and the executable
  must be linked to the TBB library.
\end{description}

These macros provide a useful mechanism for quickly reconfiguring code to
run on different devices. The following example shows a typical block of
code at the top of a source file that could be used for quick
reconfigurations.

\vtkmlisting{Macros to port VTK-m code among different devices}{DefaultDeviceAdapter.cxx}

\begin{didyouknow}
  Filters do not actually use the default device adapter tag. They have a
  more sophisticated device selection mechanism that determines the devices
  available at runtime and will attempt running on multiple devices.
\end{didyouknow}

The default device adapter can always be overridden by specifying a device
adapter tag, as described in the next section. There is actually one more
internal default device adapter named
\vtkmmacro{VTKM\_DEVICE\_ADAPTER\_ERROR} that will cause a compile error if
any component attempts to use the default device adapter. This feature is
most often used in testing code to check when device adapters should be
specified.

\subsection{Specifying Device Adapter Tags}

In addition to setting a global default device adapter, it is possible to
explicitly set which device adapter to use in any feature provided by
VTK-m. This is done by providing a device adapter tag as a template
argument to VTK-m templated objects. The following device adapter tags are
available in VTK-m.

\index{device~adapter~tag!provided|(}
\index{tag!device~adapter!provided|(}

\begin{description}
\item[\vtkmcont{DeviceAdapterTagSerial}] \index{serial} Performs all
  computation on the same single thread as the control environment. This
  device is useful for debugging. This device is always available. This tag
  is defined in \vtkmheader{vtkm/cont}{DeviceAdapterSerial.h}.
\item[\vtkmcont{DeviceAdapterTagCuda}] \index{CUDA} Uses a CUDA capable
  GPU device. For this device to work, VTK-m must be configured to use CUDA
  and the code must be compiled by the CUDA \textfilename{nvcc}
  compiler. This tag is defined in
  \vtkmheader{vtkm/cont/cuda}{DeviceAdapterCuda.h}.
\item[\vtkmcont{DeviceAdapterTagOpenMP}] \index{OpenMP} Uses OpenMP
  compiler extensions to run algorithms on multiple threads. For this
  device to work, VTK-m must be configured to use OpenMP and the code must be
  compiled with a compiler that supports OpenMP pragmas. This tag is
  defined in \vtkmheader{vtkm/openmp/cont}{DeviceAdapterOpenMP.h}. \fix{Not
    currently implemented.}
\item[\vtkmcont{DeviceAdapterTagTBB}]
  \index{Intel Threading Building Blocks} \index{TBB} Uses the Intel
  Threading Building Blocks library to run algorithms on multiple
  threads. For this device to work, VTK-m must be configured to use TBB and
  the executable must be linked to the TBB library. This tag is defined in
  \vtkmheader{vtkm/cont/tbb}{DeviceAdapterTBB.h}.
\end{description}

\index{tag!device~adapter!provided|)}
\index{device~adapter~tag!provided|)}

The following example uses the tag for the Intel Threading Building blocks
device adapter to prepare an output array for that device. In this case,
the device adapter tag is passed as a parameter for the
\textcode{PrepareForOutput} \index{PrepareForOutput} method of
\vtkmcont{ArrayHandle}.

\vtkmlisting{Specifying a device using a device adapter tag.}{SpecifyDeviceAdapter.cxx}

When structuring your code to always specify a particular device adapter,
consider setting the default device adapter (with the
\vtkmmacro{VTKM\_DEVICE\_ADAPTER} macro) to
\vtkmmacro{VTKM\_DEVICE\_ADAPTER\_ERROR}. This will cause the compiler to
produce an error if any object is instantiated with the default device
adapter, which checks to make sure the code properly specifies every device
adapter parameter.

VTK-m also defines a macro named
\vtkmmacro{VTKM\_DEFAULT\_DEVICE\_ADAPTER\_TAG}, which can be used in place
of an explicit device adapter tag to use the default tag. This macro is
used to create new templates that have template parameters for device
adapters that can use the default. The following example defines a functor
to be used with the \textcode{Schedule} operation (to be described later)
that is templated on the device it uses.

\vtkmlisting[ex:DefaultDeviceTemplateArg]{Specifying a default device for template parameters.}{DefaultDeviceTemplateArg.cxx}

\begin{commonerrors}
  A device adapter tag is a class just like every other type in C++. Thus
  it is possible to accidently use a type that is not a device adapter tag
  when one is expected. This leads to unexpected errors in strange parts of
  the code. To help identify these errors, it is good practice to use the
  \vtkmmacro{VTKM\_IS\_DEVICE\_ADAPTER\_TAG} macro to verify the type is a
  valid device adapter tag. Example~\ref{ex:DefaultDeviceTemplateArg} uses
  this macro on line 4.
\end{commonerrors}

\index{tag!device~adapter|)}
\index{device~adapter~tag|)}


\section{Device Adapter Algorithms}
\label{sec:DeviceAdapterAlgorithms}

\index{device~adapter!algorithm|(}
\index{algorithm|(}

VTK-m comes with the templated class \vtkmcont{DeviceAdapterAlgorithm} that
provides a set of algorithms that can be invoked in the control environment
and are run on the execution environment. The template has a single
argument that specifies the device adapter tag.

\vtkmlisting{Prototype for \protect\vtkmcont{DeviceAdapterAlgorithm}.}{DeviceAdapterAlgorithmPrototype.cxx}

\textidentifier{DeviceAdapterAlgorithm} contains no state. It only has a
set of static methods that implement its algorithms. The following methods
are available.

\begin{didyouknow}
  Many of the following device adapter algorithms take input and output
  \textidentifier{ArrayHandle}s, and these functions will handle their own
  memory management. This means that it is unnecessary to allocate output
  arrays. \index{Allocate} For example, it is unnecessary to call
  \textidentifier{ArrayHandle}\textcode{::Allocate()} for the output array
  passed to the \textcode{Copy} method.
\end{didyouknow}

\begin{description}
\item[\textcode{Copy}] \index{copy} Copies data from an input array to an
  output array. The copy takes place in the execution environment.
\item[\textcode{LowerBounds}] \index{lower~bounds} The
  \textcode{LowerBounds} method takes three arguments. The first argument
  is an \textidentifier{ArrayHandle} of sorted values. The second argument
  is another \textidentifier{ArrayHandle} of items to find in the first
  array. \textcode{LowerBounds} find the index of the first item that is
  greater than or equal to the target value, much like the
  \textcode{std::lower\_bound} STL algorithm. The results are returned in
  an \textidentifier{ArrayHandle} given in the third argument.

  There are two specializations of \textcode{LowerBounds}. The first takes
  an additional comparison function that defines the less-than
  operation. The second takes only two parameters. The first is an
  \textidentifier{ArrayHandle} of sorted \vtkm{Id}s and the second is an
  \textidentifier{ArrayHandle} of \vtkm{Id}s to find in the first list. The
  results are written back out to the second array. This second
  specialization is useful for inverting index maps.
\item[\textcode{Reduce}] \index{reduce} The \textcode{Reduce} method takes
  an input array, initial value, and a binary function and computes a
  ``total'' of applying the binary function to all entries in the array.
  The provided binary function must be associative (but it need not be
  commutative). There is a specialization of \textcode{Reduce} that does
  not take a binary function and computes the sum.
\item[\textcode{ReduceByKey}] \index{reduce~by~key} The
  \textcode{ReduceByKey} method works similarly to the \textcode{Reduce}
  method except that it takes an additional array of keys, which must be
  the same length as the values being reduced. The arrays are partitioned
  into segments that have identical adjacent keys, and a separate reduction
  is performed on each partition. The unique keys and reduced values are
  returned in separate arrays.
\item[\textcode{ScanInclusive}] \index{scan!inclusive} The
  \textcode{ScanInclusive} method takes an input and an output
  \textidentifier{ArrayHandle} and performs a running sum on the input
  array. The first value in the output is the same as the first value in
  the input. The second value in the output is the sum of the first two
  values in the input. The third value in the output is the sum of the
  first three values of the input, and so on. \textcode{ScanInclusive}
  returns the sum of all values in the input. There are two forms of
  \textcode{ScanInclusive}: one performs the sum using addition whereas the
  other accepts a custom binary function to use as the sum operator.
\item[\textcode{ScanExclusive}] \index{scan!exclusive} The
  \textcode{ScanExclusive} method takes an input and an output
  \textidentifier{ArrayHandle} and performs a running sum on the input
  array. The first value in the output is always 0. The second value in the
  output is the same as the first value in the input. The third value in
  the output is the sum of the first two values in the input. The fourth
  value in the output is the sum of the first three values of the input,
  and so on. \textcode{ScanExclusive} returns the sum of all values in the
  input. There are two forms of \textcode{ScanExclusive}: one performs the
  sum using addition whereas the other accepts a custom binary function to
  use as the sum operator and a custom initial value to use in the running
  sum.
\item[\textcode{Schedule}] \index{schedule} The \textcode{Schedule} method
  takes a functor as its first argument and invokes it a number of times
  specified by the second argument. It should be assumed that each
  invocation of the functor occurs on a separate thread although in
  practice there could be some thread sharing.

  There are two versions of the \textcode{Schedule} method. The first
  version takes a \vtkm{Id} and invokes the functor that number of
  times. The second version takes a \vtkm{Id3} and invokes the functor once
  for every entry in a 3D array of the given dimensions.

  The functor is expected to be an object with a const overloaded
  parentheses operator. The operator takes as a parameter the index of the
  invocation, which is either a \vtkm{Id} or a \vtkm{Id3} depending on what
  version of \textcode{Schedule} is being used. The functor must also
  subclass \vtkmexec{FunctorBase}, which provides the error handling
  facilities for the execution environment. \textidentifier{FunctorBase}
  contains a public method named \index{RaiseError}
  \index{errors!execution~environment} \textcode{RaiseError} that takes a
  message and will cause a \vtkmcont{ErrorExecution} exception to be thrown
  in the control environment.
\item[\textcode{Sort}] \index{sort} The \textcode{Sort} method provides an
  unstable sort of an array. There are two forms of the \textcode{Sort}
  method. The first takes an \textidentifier{ArrayHandle} and sorts the
  values in place. The second takes an additional argument that is a
  functor that provides the comparison operation for the sort.
\item[\textcode{SortByKey}] \index{sort!by key} The \textcode{SortByKey}
  method works similarly to the \textcode{Sort} method except that it takes
  two \textidentifier{ArrayHandle}s: an array of keys and a corresponding
  array of values. The sort orders the array of keys in ascending values
  and also reorders the values so they remain paired with the same
  key. Like \textcode{Sort}, \textcode{SortByKey} has a version that sorts
  by the default less-than operator and a version that accepts a custom
  comparison functor.
\item[\textcode{StreamCompact}] \index{stream~compact} The
  \textcode{StreamCompact} method selectively removes values from an
  array. The first argument is an \textidentifier{ArrayHandle} to be
  compacted and the second argument is an \textidentifier{ArrayHandle} of
  equal size with flags indicating whether the corresponding input value is
  to be copied to the output. The third argument is an output
  \textidentifier{ArrayHandle} whose length is set to the number of true
  flags in the stencil and the passed values are put in order to the output
  array.

  There is also a second form of \textidentifier{StreamCompact} that only
  has the stencil and output as arguments. In this version, the output gets
  the corresponding index of where the input should be taken from.
\item[\textcode{Synchronize}] \index{synchronize} The
  \textidentifier{Synchronize} method waits for any asynchronous operations
  running on the device to complete and then returns.
\item[\textcode{Unique}] \index{unique} The \textcode{Unique} method
  removes all duplicate values in an \textidentifier{ArrayHandle}. The
  method will only find duplicates if they are adjacent to each other in
  the array. The easiest way to ensure that duplicate values are adjacent
  is to sort the array first.

  There are two versions of \textcode{Unique}. The first uses the equals
  operator to compare entries. The second accepts a binary functor to
  perform the comparisons.
\item[\textcode{UpperBounds}] \index{upper~bounds} The
  \textcode{UpperBounds} method takes three arguments. The first argument
  is an \textidentifier{ArrayHandle} of sorted values. The second argument
  is another \textidentifier{ArrayHandle} of items to find in the first
  array. \textcode{UpperBounds} find the index of the first item that is
  greater than to the target value, much like the
  \textcode{std::upper\_bound} STL algorithm. The results are returned in
  an \textidentifier{ArrayHandle} given in the third argument.

  There are two specializations of \textcode{UpperBounds}. The first takes
  an additional comparison function that defines the less-than
  operation. The second takes only two parameters. The first is an
  \textidentifier{ArrayHandle} of sorted \vtkm{Id}s and the second is an
  \textidentifier{ArrayHandle} of \vtkm{Id}s to find in the first list. The
  results are written back out to the second array. This second
  specialization is useful for inverting index maps.
\end{description}

\index{algorithm|)}
\index{device~adapter!algorithm|)}


\section{Implementing Device Adapters}
\label{sec:ImplementingDeviceAdapters}

VTK-m comes with several implementations of device adapters so that it may
be ported to a variety of platforms. It is also possible to provide new
device adapters to support yet more devices, compilers, and libraries. A
new device adapter provides a tag, a class to manage arrays in the
execution environment, a collection of algorithms that run in the execution
environment, and (optionally) a timer.

Most device adapters are associated with some type of device or library,
and all source code related directly to that device is placed in a
subdirectory of \textfilename{vtkm/cont}. For example, files associated
with CUDA are in \textfilename{vtkm/cont/cuda} and files associated with
the Intel Threading Building Blocks (TBB) are located in
\textfilename{vtkm/cont/tbb}. The documentation here assumes that you are
adding a device adapter to the VTK-m source code and following these file
conventions. However, it is also possible to define a device adapter
outside of the core VTK-m, in which case the file paths might be different.

For the purposes of discussion in this section, we will give a simple
example of implementing a device adapter using the \textcode{std::thread}
class provided by C++11. We will call our device \textcode{Cxx11Thread} and
place it in the directory \textfilename{vtkm/cont/cxx11}.

By convention the implementation of device adapters within VTK-m are
divided into 3 header files with the names
\textfilename{DeviceAdapterTag\textasteriskcentered.h},
\textfilename{ArrayManagerExecution\textasteriskcentered.h} and
\textfilename{DeviceAdapterAlgorithm\textasteriskcentered.h}, which are
hidden in internal directories. The
\textfilename{DeviceAdapter\textasteriskcentered.h} that most code includes
is a trivial header that simply includes these other three files. For our
example \textcode{std::thread} device, we will create the base header at
\textfilename{vtkm/cont/cxx11/DeviceAdapterCxx11Thread.h}. The contents are
the following (with minutia like include guards removed).

\begin{vtkmexample}{Contents of the base header for a device adapter.}
#include <vtkm/cont/cxx11/internal/DeviceAdapterTagCxx11Thread.h>
#include <vtkm/cont/cxx11/internal/ArrayManagerExecutionCxx11Thread.h>
#include <vtkm/cont/cxx11/internal/DeviceAdapterAlgorithmCxx11Thread.h>
\end{vtkmexample}

The reason VTK-m breaks up the code for its device adapters this way is
that there is an interdependence between the implementation of each device
adapter and the mechanism to pick a default device adapter. Breaking up the
device adapter code in this way maintains an acyclic dependence among
header files.

\subsection{Tag}

\index{device~adapter!tag|(}

The device adapter tag, as described in Section~\ref{sec:DeviceAdapterTag}
is a simple empty type that is used as a template parameter to identify the
device adapter. Every device adapter implementation provides one. The
device adapter tag is typically defined in an internal header file with a
prefix of \textfilename{DeviceAdapterTag}.

The device adapter tag should be created with the macro
\vtkmmacro{VTKM\_VALID\_DEVICE\_ADAPTER}. This adapter takes an abbreviated
name that it will append to \textcode{DeviceAdapterTag} to make the tag
structure. It will also create some support classes that allow VTK-m to
introspect the device adapter. The macro also expects a unique integer
identifier that is usually stored in a macro prefixed with
\textcode{VTKM\_DEVICE\_ADAPTER\_}. These identifiers for the device
adapters provided by the core VTK-m are declared in
\vtkmheader{vtkm/cont/internal}{DeviceAdapterTag.h}.

The following example gives the implementation of our custom device
adapter, which by convention would be placed in the
\textfilename{vtkm/cont/cxx11/internal/DeviceAdapterTagCxx11Thread.h}
header file.

\vtkmlisting{Implementation of a device adapter tag.}{DeviceAdapterTagCxx11Thread.h}

\index{device~adapter!tag|)}

\subsection{Array Manager Execution}

\index{device~adapter!array manager|(}
\index{array~manager~execution|(}
\index{execution~array~manager|(}

VTK-m defines a template named \vtkmcontinternal{ArrayManagerExecution}
that is responsible for allocating memory in the execution environment and
copying data between the control and execution environment. The execution
array manager is typically defined in an internal header file with a prefix
of \textfilename{ArrayManagerExecution}.

\vtkmlisting{Prototype for \protect\vtkmcontinternal{ArrayManagerExecution}.}{ArrayManagerExecutionPrototype.cxx}

A device adapter must provide a partial specialization of
\textidentifier{ArrayManagerExecution} for its device adapter tag. The
implementation for \textidentifier{ArrayManagerExecution} is expected to
manage the resources for a single array. All
\textidentifier{ArrayManagerExecution} specializations must have a
constructor that takes a pointer to a \vtkmcontinternal{Storage} object.
The \textidentifier{ArrayManagerExecution} should store a reference to this
\textidentifier{Storage} object and use it to pass data between control and
execution environments. Additionally,
\textidentifier{ArrayManagerExecution} must provide the following elements.

\begin{description}
\item[\textcode{ValueType}] A \textcode{typedef} of the type for each item
  in the array. This is the same type as the first template argument.
\item[\textcode{PortalType}] The type of an array portal that can be used
  in the execution environment to access the array.
\item[\textcode{PortalConstType}] A read-only (const) version of
  \textcode{PortalType}.
\item[\textcode{GetNumberOfValues}] A method that returns the number of
  values stored in the array. The results are undefined if the data has not
  been loaded or allocated.
\item[\textcode{PrepareForInput}] A method that ensures an array is
  allocated in the execution environment and valid data is there. The
  method takes a \textcode{bool} flag that specifies whether data needs to
  be copied to the execution environment. (If false, then data for this
  array has not changed since the last operation.) The method returns a
  \textcode{PortalConstType} that points to the data.
\item[\textcode{PrepareForInPlace}] A method that ensures an array is
  allocated in the execution environment and valid data is there. The
  method takes a \textcode{bool} flag that specifies whether data needs to
  be copied to the execution environment. (If false, then data for this
  array has not changed since the last operation.) The method returns a
  \textcode{PortalType} that points to the data.
\item[\textcode{PrepareForOutput}] A method that takes an array
  size and allocates an array in the execution environment
  of the specified size. The initial memory may be uninitialized. The
  method returns a \textcode{PortalType} to the data.
\item[\textcode{RetrieveOutputData}] This method takes a storage object,
  allocates memory in the control environment, and copies data from the
  execution environment into it. If the control and execution environments
  share arrays, then this can be a no-operation.
\item[\textcode{CopyInto}] This method takes an STL-compatible iterator and
  copies data from the execution environment into it.
\item[\textcode{Shrink}] A method that adjusts the size of the array in the
  execution environment to something that is a smaller size. All the data
  up to the new length must remain valid. Typically, no memory is actually
  reallocated. Instead, a different end is marked.
\item[\textcode{ReleaseResources}] A method that frees any resources
  (typically memory) in the execution environment.
\end{description}

Specializations of this template typically take on one of two forms. If the
control and execution environments have separate memory spaces, then this
class behaves by copying memory in methods such as
\textcode{PrepareForInput} and \textcode{RetrieveOutputData}. This might
require creating buffers in the control environment to efficiently move
data from control array portals.

However, if the control and execution environments share the same memory
space, the execution array manager can, and should, delegate all of its
operations to the \textidentifier{Storage} it is constructed with. VTK-m
comes with a class called
\vtkmcontinternal{ArrayManagerExecutionShareWithControl} that provides the
implementation for an execution array manager that shares a memory space
with the control environment. In this case, making the
\textidentifier{ArrayManagerExecution} specialization be a trivial subclass
is sufficient. Continuing our example of a device adapter based on C++11's
\textcode{std::thread} class, here is the implementation of
\textidentifier{ArrayManagerExecution}, which by convention would be placed
in the
\textfilename{vtkm/cont/cxx11/internal/ArrayManagerExecutionCxx11Thread.h}
header file.

\vtkmlisting{Specialization of \textidentifier{ArrayManagerExecution}.}{ArrayManagerExecutionCxx11Thread.h}

\index{execution~array~manager|)}
\index{array~manager~execution|)}
\index{device~adapter!array manager|)}

\subsection{Algorithms}

\index{device~adapter!algorithm|(}
\index{algorithm|(}

A device adapter implementation must also provide a specialization of
\vtkmcont{DeviceAdapterAlgorithm}, which is documented in
Section~\ref{sec:DeviceAdapterAlgorithms}. The implementation for the
device adapter algorithms is typically placed in a header file with a
prefix of \textfilename{DeviceAdapterAlgorithm}.

Although there are many methods in \textidentifier{DeviceAdapterAlgorithm},
it is seldom necessary to implement them all. Instead, VTK-m comes with
\vtkmcontinternal{DeviceAdapterAlgorithmGeneral} that provides generic
implementation for most of the required algorithms. By deriving the
specialization of \textidentifier{DeviceAdapterAlgorithm} from
\textidentifier{DeviceAdapterAlgorithmGeneral}, only the implementations
for \textcode{Schedule} and \textcode{Synchronize} need to be implemented.
All other algorithms can be derived from those.

That said, not all of the algorithms implemented in
\textidentifier{DeviceAdapterAlgorithmGeneral} are optimized for all types
of devices. Thus, it is worthwhile to provide algorithms optimized for the
specific device when possible. In particular, it is best to provide
specializations for the sort, scan, and reduce algorithms.

One point to note when implementing the \textcode{Schedule} methods is to
make sure that errors handled in the execution environment are handled
correctly. As described in
Section~\ref{sec:ExecutionEnvironment:ErrorHandling}, errors are signaled
in the execution environment by calling \textcode{RaiseError} on a functor
or worklet object. This is handled internally by the
\vtkmexecinternal{ErrorMessageBuffer} class.
\textidentifier{ErrorMessageBuffer} really just holds a small string
buffer, which must be provided by the device adapter's \textcode{Schedule}
method.

So, before \textcode{Schedule} executes the functor it is given, it should
allocate a small string array in the execution environment, initialize it
to the empty string, encapsulate the array in an
\textidentifier{ErrorMessageBuffer} object, and set this buffer object in
the functor. When the execution completes, \textcode{Schedule} should check
to see if an error exists in this buffer and throw a
\vtkmcont{ErrorExecution} if an error has been reported.

\begin{commonerrors}
  Exceptions are generally not supposed to be thrown in the execution
  environment, but it could happen on devices that support them.
  Nevertheless, few thread schedulers work well when an exception is thrown
  in them. Thus, when implementing adapters for devices that do support
  exceptions, it is good practice to catch them within the thread and
  report them through the \textidentifier{ErrorMessageBuffer}.
\end{commonerrors}

The following example is a minimal implementation of device adapter
algorithms using C++11's \textcode{std::thread} class. Note that no attempt
at providing optimizations has been attempted (and many are possible). By
convention this code would be placed in the
\textfilename{vtkm/cont/cxx11/internal/DeviceAdapterAlgorithmCxx11Thread.h}
header file.

\vtkmlisting{Minimal specialization of \textidentifier{DeviceAdapterAlgorithm}.}{DeviceAdapterAlgorithmCxx11Thread.h}

\index{algorithm|)}
\index{device~adapter!algorithm|)}

\subsection{Timer Implementation}

\index{timer|(}
\index{device~adapter!timer|(}

The VTK-m timer, described in Chapter~\ref{chap:Timers}, delegates to an
internal class named \vtkmcont{DeviceAdapterTimerImplementation}. The
interface for this class is the same as that for \vtkmcont{Timer}. A default
implementation of this templated class uses the system timer and the
\textcode{Synchronize} method in the device adapter algorithms.

However, some devices might provide alternate or better methods for
implementing timers. For example, the TBB and CUDA libraries come with high
resolution timers that have better accuracy than the standard system
timers. Thus, the device adapter can optionally provide a specialization of
\textidentifier{DeviceAdapterTimerImplementation}, which is typically
placed in the same header file as the device adapter algorithms.

Continuing our example of a custom device adapter using C++11's
\textcode{std::thread} class, we could use the default timer and it would
work fine. But C++11 also comes with a \textcode{std::chrono} package that
contains some portable time functions. The following code demonstrates
creating a custom timer for our device adapter using this package. By
convention, \textidentifier{DeviceAdapterTimerImplementation} is placed in
the same header file as \textidentifier{DeviceAdapterAlgorithm}.

\vtkmlisting{Specialization of \textidentifier{DeviceAdapterTimerImplementation}.}{DeviceAdapterTimerImplementationCxx11Thread.h}

\index{device~adapter!timer|)}
\index{timer|)}


\index{device~adapter|)}
