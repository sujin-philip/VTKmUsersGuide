% -*- latex -*-

\chapter{Execution Environment}
\label{chap:ExecutionEnvironment}

The execution environment is exposed to developers that write worklets for
different visualization algorithms. In addition to providing all the
mechanisms for building the worklet object itself, the execution
environment contains supporting code that can be useful to the
implementations of visualization algorithms.

The data structures in the execution environment provide information and
operations for a single element. This is in contrast to the control
environment, where data structures are built on arrays providing
information for large collections of data. These respective data structures
reflect the nature of the two environments. The control environment manages
the stores of data whereas the execution environment performs large
parallel processing through fine operations.


\section{Error Handling}
\label{sec:ExecutionEnvironment:ErrorHandling}

\section{Math}


\section{Working with Topology}

In the control environment, data is defined in mesh structures that
comprise a set of finite cells. (See Section~\ref{sec:DataSets:CellSets}
starting on page~\pageref{sec:DataSets:CellSets} for information on
defining cell sets in the control environment.) When worklets that operate
on cells are scheduled, these grid structures are broken into their
independent cells, and that data is handed to the worklet. Thus, cell-based
operations in the execution environment exclusively operate on independent
cells.

Unlike some other libraries such as VTK, VTK-m does not have a cell class
that holds all the information pertaining to a cell of a particular type.
Instead, VTK-m provides tags or identifiers defining the cell shape, and
companion data like coordinate and field information are held in separate
structures. This organization is designed so a worklet may specify exactly
what information it needs, and only that information will be loaded.

\subsection{Cell Shape Tags and Ids}
\label{sec:CellShapeTagsIds}

\index{shape|(}
\index{cell~shape|(}
\index{tag!cell~shape|(}
\index{tag!shape|(}

Cell shapes can be specified with either a tag (defined with a struct with
a name like \textidentifier{CellShapeTag*}) or an enumerated identifier
(defined with a constant number with a name like
\textidentifier{CELL\_SHAPE\_*}). These shape tags and identifiers are
defined in \vtkmheader{vtkm}{CellShape.h} and declared in the \vtkm{}
namespace (because they can be used in either the control or the execution
environment). Figure~\ref{fig:CellShapes} gives both the identifier and the
tag names.

\begin{figure}
  \centering
  \small
  \begin{tabular}{@{}c@{~}c@{~}c@{}}
    \raisebox{-0.5\height}{\includegraphics{images/CellConnectionsVertex}} &
    \raisebox{-0.5\height}{\includegraphics{images/CellConnectionsLine}} &
    \raisebox{-0.5\height}{\includegraphics{images/CellConnectionsTriangle}} \\
    \vtkm{CELL\_SHAPE\_VERTEX} &
    \vtkm{CELL\_SHAPE\_LINE} &
    \vtkm{CELL\_SHAPE\_TRIANGLE} \\
    \vtkm{CellShapeTagVertex} \index{vertex} &
    \vtkm{CellShapeTagLine} \index{line} &
    \vtkm{CellShapeTagTriangle} \index{triangle} \\[2ex]
    \raisebox{-0.5\height}{\includegraphics{images/CellConnectionsPolygon}} &
    \raisebox{-0.5\height}{\includegraphics{images/CellConnectionsQuadrilateral}} &
    \raisebox{-0.5\height}{\includegraphics{images/CellConnectionsTetrahedron}} \\
    \vtkm{CELL\_SHAPE\_POLYGON} &
    \vtkm{CELL\_SHAPE\_QUAD} &
    \vtkm{CELL\_SHAPE\_TETRA} \\
    \vtkm{CellShapeTagPolygon} \index{polygon} &
    \vtkm{CellShapeTagQuad} \index{quadrilateral} &
    \vtkm{CellShapeTagTetra} \index{tetrahedron} \\[2ex]
    \raisebox{-0.5\height}{\includegraphics{images/CellConnectionsHexahedron}} &
    \raisebox{-0.5\height}{\includegraphics{images/CellConnectionsWedge}} &
    \raisebox{-0.5\height}{\includegraphics{images/CellConnectionsPyramid}} \\
    \vtkm{CELL\_SHAPE\_HEXAHEDRON} &
    \vtkm{CELL\_SHAPE\_WEDGE} &
    \vtkm{CELL\_SHAPE\_PYRAMID} \\
    \vtkm{CellShapeTagHexahedron} \index{hexahedron} &
    \vtkm{CellShapeTagWedge} \index{wedge} &
    \vtkm{CellShapeTagPyramid} \index{pyramid}
  \end{tabular}
  \caption{Basic Cell Shapes}
  \label{fig:CellShapes}
\end{figure}

In addition to the basic cell shapes, there is a special ``empty'' cell
with the identifier \vtkm{CELL\_SHAPE\_EMPTY} and tag
\vtkm{CellShapeTagEmpty}. This type of cell has no points, edges, or faces
and can be thought of as a placeholder for a null or void cell.

There is also a special cell shape ``tag'' named \vtkm{CellShapeTagGeneric}
that is used when the actual cell shape is not known at compile time.
\textidentifier{CellShapeTagGeneric} actually has a member variable named
\textcode{Id} that stores the identifier for the cell shape. There is no
equivalent identifier for a generic cell; cell shape identifiers can be
placed in a \vtkm{IdComponent} at runtime.

\fix{Add other basic cell shape features such as traits, converting back
  and forth, and \vtkmmacro{vtkmGenericCellShapeMacro}.}

\index{tag!shape|)}
\index{tag!cell~shape|)}
\index{cell~shape|)}
\index{shape|)}

\subsection{Parametric and World Coordinates}

\subsection{Interpolation}

\subsection{Derivatives}
