% -*- latex -*-

\chapter{Introduction}
\label{chap:Introduction}

\fix{Write this. Combination of chapters 1 and 2 in the Dax report plus
  added features like datasets and filters.}

\fix{Also include the following.}

VTK-m is written in C++ and makes extensive use of templates. The toolkit
is implemented as a header library, meaning that all the code is
implemented in header files (with extension \textfilename{.h}) and
completely included in any code that uses it. \fix{Verify that does not
  change by version 1.0} This is typically necessary of template libraries,
which need to be compiled with template parameters that are not known until
they are used. This also provides the convenience of allowing the compiler
to inline user code for better performance.

When documenting the VTK-m API, the following conventions are used.
\begin{itemize}
\item Filenames are printed in a \textfilename{sans serif font}.
\item C++ code is printed in a \textcode{monospace font}.
\item Macros and namespaces from the Dax toolkit are printed
  in \textnamespace{red}.
\item Identifiers from the Dax toolkit are printed in
  \textidentifier{blue}.
\item Signatures, described in Section~\ref{sec:GenericScheduling}, and the
  tags used in them are printed in \textsignature{green}.
\end{itemize}
