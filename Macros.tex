% Collection of macros and stuff to use in the user-guide.

%-------------------------------------------------------------------------------
% stuff for "boxed text" used for didyouknow and common error boxes.
%-------------------------------------------------------------------------------
\usepackage[tikz]{bclogo}
\usepackage[framemethod=tikz]{mdframed}
\usepackage{ifthen}

\definecolor{bgblue}{RGB}{245,243,253}
\definecolor{ttblue}{RGB}{91,194,224}
\definecolor{bgyellow}{RGB}{255,255,204}


% change title color.
\renewcommand\bcStyleTitre[1]{\large\textcolor{ttblue}{#1}}
%\begin{bclogo}[couleur=bgblue, arrondi =0 , logo=\bcbombe, barre=none,noborder=true]{Commom Programming Error}
%\itshape\lipsum[4]
%\end{bclogo}

\newenvironment{commonerrors}%
{\vspace{1em}\begin{bclogo}[couleur=bgblue, arrondi =0 , logo=\bcbombe, barre=snake,noborder=true]{Common Errors}\itshape}%
{\end{bclogo}\vspace{1em}}

\newenvironment{didyouknow}%
{\vspace{1em}\begin{bclogo}[couleur=bgblue, arrondi =0 , logo=\bcinfo, barre=snake,noborder=true]{Did you know?}\itshape}%
{\end{bclogo}\vspace{1em}}


% Cite commands I use to abstract away the different ways to reference an
% entry in the bibliography (superscripts, numbers, dates, or author
% abbreviations).  \scite is a short cite that is used immediately after
% when the authors are mentioned.  \lcite is a full citation that is used
% anywhere.  Both should be used right next to the text being cited without
% any spacing.
\newcommand*{\lcite}[1]{~\cite{#1}}
\newcommand*{\scite}[1]{~\cite{#1}}

\newcommand{\etal}{et al.}

\newcommand*{\keyterm}[1]{\emph{#1}}
