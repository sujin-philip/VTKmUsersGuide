% -*- latex -*-

\chapter{Provided Filters}
\label{chap:ProvidedFilters}

\index{filter|(}

Filters are functional units that take data as input and write new data as
output. Filters operate on \vtkmcont{DataSet} objects, which are introduced
with the file I/O operations in Chapter~\ref{chap:IO} and are described in
more detail in Chapter~\ref{chap:DataSet}. For now we treat
\textidentifier{DataSet} mostly as an opaque object that can be passed
around readers, writers, filters, and rendering units.

\begin{didyouknow}
  The structure of filters in VTK-m is significantly simpler than their
  counterparts in VTK. VTK filters are arranged in a dataflow network
  (a.k.a. a visualization pipeline) and execution management is handled
  automatically. In contrast, VTK-m filters are simple imperative units,
  which are simply called with input data and return output data.
\end{didyouknow}

VTK-m comes with several filters ready for use, and in this chapter we will
give a brief overview of these filters. We group filters based on the type
of operation that they do and the shared interfaces that they have. Later
Part~\ref{part::Developing} describes the necessary steps in creating new
filters in VTK-m.


\section{Field Filters}

\index{filter!field|(}
\index{field~filter|(}

Every \vtkmcont{DataSet} object contains a list of \index{field}
\keyterm{fields}. A field describes some numerical value associated with
different parts of the data set in space. Fields often represent physical
properties such as temperature, pressure, or velocity. \keyterm{Field
  filters} are a class of filters that generate a new field. These new
fields are typically derived from one or more existing fields or point
coordinates on the data set. For example, mass, volume, and density are
interrelated, and any one can be derived from the other two.

All field filters contain an \textcode{Execute} method that takes two
arguments. The first argument is a \vtkmcont{DataSet} object with the input
data. The second argument specifies the field from which to derive a new
field. The field can be specified as either a string naming a field in the
input \textidentifier{DataSet} object, as a \vtkmcont{Field} object, or as
a coordinate system (typically retrived from a \textidentifier{DataSet}
object with the \textcode{GetCoordianteSystem} method). See Sections
\ref{sec:DataSets:Fields} and \ref{sec:DataSets:CoordinateSystems} for more
information on fields and coordinate systems, respectively.

Field filters often need more information than just a data set and a field.
Any additional information is provided using methods in the filter class
that changes the state. These methods are called before \textcode{Execute}.
One such method that all field filters has is
\textcode{SetOutputFieldName}, which specifies the name assigned to the
generated field. If not specified, then the filter will use a default name.

The \textcode{Execute} method returns a \vtkmfilter{ResultField} object,
which contains the state of the execution and the data generated. A
\textidentifier{ResultField} object has the following methods.

\begin{description}
\item[\textcode{IsValid}] Returns a \textcode{bool} value specifying
  whether the execution completed successfully. If \textcode{true}, then
  the execution was successful and the data stored in the
  \textidentifier{ResultField} is valid. If \textcode{false}, then the
  execution failed.
\item[\textcode{GetDataSet}] Returns a \textidentifier{DataSet} containing
  the results of the execution. The data set returned is a shallow copy of
  the input data with the generated field added.
\item[\textcode{GetField}] Returns the field information in a
  \vtkmcont{Field} object. Field objects are described in
  Section~\ref{sec:DataSets:Fields}.
\item[\textcode{FieldAs}] Given a \vtkmcont{ArrayHandle} object, allocates
  the array and copies the generated field data into it.
\end{description}

The following example provides a simple demonstration of using a field
filter. It specifically uses the point elevation filter, which is one of
the field filters.

\vtkmlisting[ex:PointElevation]{Using \textidentifier{PointElevation}, which is a field filter.}{PointElevation.cxx}

\subsection{Cell Average}

\index{cell~average|(}
\index{average|(}

\vtkmfilter{CellAverage} is the cell average filter. It will take a data
set with a collection of cells and a field defined on the points of the
data set and create a new field defined on the cells. The values of this
new derived field are computed by averaging the values of the input field
at all the incident points. This is a simple way to convert a point field
to a cell field. Both the input data set and the input field are specified
as arguments to the \textcode{Execute} method.

The default name for the output cell field is the same name as the input
point field. The name can be overridden using the
\textcode{SetOutputFieldName} method.

In addition the standard \textcode{SetOutputFieldName} and
\textcode{Execute} methods, \textidentifier{CellAverage} provides the
following methods.

\begin{description}
\item[\textcode{SetActiveCellSet}] Sets the index for the cell set to use
  from the \textidentifier{DataSet} provided to the \textcode{Execute}
  method. The default index is 0, which is the first cell set. If you want
  to set the active cell set by name, you can use the
  \textidentifier{GetCellSetIndex} method on the \textidentifier{DataSet}
  object that will be used with \textcode{Execute}.
\item[\textcode{GetActiveCellSetIndex}] Returns the index to be used when
  getting a cell set from the \textidentifier{DataSet} passed to
  \textcode{Execute}. Set with \textcode{SetActiveCellSet}.
\end{description}

\index{average|}
\index{cell~average|)}

\subsection{Point Elevation}

\index{point~elevation|(}
\index{elevation|(}

\vtkmfilter{PointElevation} computes the ``elevation'' of a field of point
coordinates in space. The filter will take a data set and a field of 3
dimensional vectors and compute the distance along a line defined by a low
point and a high point. Any point in the plane touching the low point and
perpendicular to the line is set to the minimum range value in the
elevation whereas any point in the plane touching the high point and
perpendicular to the line is set to the maximum range value. All other
values are interpolated linearly between these two planes. This filter is
commonly used to compute the elevation of points in some direction, but can
be repurposed for a variety of measures.

The input field (or coordinate system) is specified as the second argument
to the \textcode{Execute} method. A \vtkmcont{DataSet} that is expected to
contain the field is also given but is otherwise unused.
Example~\ref{ex:PointElevation} gives a demonstration of the elevation
filter.

The default name for the output field is ``elevation'', but that can be
overridden using the \textcode{SetOutputFieldName} method.

In addition to the standard \textcode{SetOutputFieldName} and
\textcode{Execute} methods, \textidentifier{PointElevation} provides the
following methods.

\begin{description}
\item[\textcode{SetLowPoint}/\textcode{SetHighPoint}] This pair of methods
  is used to set the low and high points, respectively, of the elevation.
  Each method takes three floating point numbers specifying the $x$, $y$,
  and $z$ components of the low or high point.
\item[\textcode{SetRange}] Sets the range of values to use for the output
  field. This method takes two floating point numbers specifying the low
  and high values, respectively.
\end{description}

\index{elevation|)}
\index{point~elevation|)}

\index{field~filter|)}
\index{filter!field|)}


\section{Data Set Filters}

\index{filter!data~set|(}
\index{data~set~filter|(}

\keyterm{Data set filters} are a class of filters that generate a new data
set with a new topology. This new topology is typically derived from an
existing data set. For example, a data set can be significantly alterned by
adding, removing, or replacing cells.

All data set filters contain an \textcode{Execute} method that takes one
argument: a \vtkmcont{DataSet} object with the input data. The
\textcode{Execute} method returns a \vtkmfilter{ResultDataSet} object,
which contains the state of the execution and the data generated. A
\textidentifier{ResultDataSet} object has the following methods.

\begin{description}
\item[\textcode{IsValid}] Returns a \textcode{bool} value specifying
  whether the execution completed successfully. If \textcode{true}, then
  the execution was successful and the data stored in the
  \textidentifier{ResultField} is valid. If \textcode{false}, then the
  execution failed.
\item[\textcode{GetDataSet}] Returns a \textidentifier{DataSet} containing
  the results of the execution.
\end{description}

Some data set filters need more information that just a data set when
executing. Any additional information is provided using methods in the
filter class that changes the state. These methods are called before
\textcode{Execute}.

Because the new data set is derived from existing data, it can often
inherit field information from the original data. All data set filters also
contain a \textcode{MapFieldOntoOutput} method to map fields from the
output to the input. This method takes two arguments. The first argument is
the \textidentifier{ResultDataSet} object returned from the last call to
\textcode{Execute}. The second argument is a \vtkmcont{Field} object that
comes from the input. \textcode{MapFieldOntoOutput} returns a
\textcode{bool} that is true if the field was successfully mapped and added
to the output data set in the \textidentifier{ResultDataSet} object.

The following example provides a simple demonstration of using a data set
filter. It specifically uses the vertex clustring filter, which is one of
the data set filters.

\vtkmlisting[ex:VertexClustering]{Using \textidentifier{VertexClustering}, which is a data set filter.}{VertexClustering.cxx}

\index{data~set~filter|)}
\index{filter!data~set|)}


\section{Data Set and Field Filters}

\index{filter!data~set~with~field|(}
\index{data~set~with~filter|(}

\index{data~set~with~filter|)}
\index{filter!data~set~with~field|)}


\index{filter|)}
