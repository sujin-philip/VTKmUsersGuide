% -*- latex -*-

\chapter{Provided Filters}
\label{chap:ProvidedFilters}

\index{filter|(}

Filters are functional units that take data as input and write new data as
output. Filters operate on \vtkmcont{DataSet} objects, which are introduced
with the file I/O operations in Chapter~\ref{chap:IO} and are described in
more detail in Chapter~\ref{chap:DataSet}. For now we treat
\textidentifier{DataSet} mostly as an opaque object that can be passed
around readers, writers, filters, and rendering units.

\begin{didyouknow}
  The structure of filters in VTK-m is significantly simpler than their
  counterparts in VTK. VTK filters are arranged in a dataflow network
  (a.k.a. a visualization pipeline) and execution management is handled
  automatically. In contrast, VTK-m filters are simple imperative units,
  which are simply called with input data and return output data.
\end{didyouknow}

VTK-m comes with several filters ready for use, and in this chapter we will
give a brief overview of these filters. We group filters based on the type
of operation that they do and the shared interfaces that they have. Later
Part~\ref{part::Developing} describes the necessary steps in creating new
filters in VTK-m.


\section{Field Filters}

\index{filter!field|(}
\index{field~filter|(}

Every \vtkmcont{DataSet} object contains a list of \index{field}
\keyterm{fields}. A field describes some numerical value associated with
different parts of the data set in space. Fields often represent physical
properties such as temperature, pressure, or velocity. \keyterm{Field
  filters} are a class of filters that generate a new field. These new
fields are typically derived from one or more existing fields or point
coordinates on the data set. For example, mass, volume, and density are
interrelated, and any one can be derived from the other two.

All field filters contain an \textcode{Execute} method that takes two
arguments. The first argument is a \vtkmcont{DataSet} object with the input
data. The second argument specifies the field from which to derive a new
field. The field can be specified as either a string naming a field in the
input \textidentifier{DataSet} object, as a \vtkmcont{Field} object, or as
a coordinate system (typically retrived from a \textidentifier{DataSet}
object with the \textcode{GetCoordianteSystem} method). See Sections
\ref{sec:DataSets:Fields} and \ref{sec:DataSets:CoordinateSystems} for more
information on fields and coordinate systems, respectively.

Field filters often need more information than just a data set and a field.
Any additional information is provided using methods in the filter class
that changes the state. These methods are called before \textcode{Execute}.
One such method that all field filters has is
\textcode{SetOutputFieldName}, which specifies the name assigned to the
generated field. If not specified, then the filter will use a default name.

The \textcode{Execute} method returns a \vtkmfilter{ResultField} object,
which contains the state of the execution and the data generated. A
\textidentifier{ResultField} object has the following methods.

\begin{description}
\item[\textcode{IsValid}] Returns a \textcode{bool} value specifying
  whether the execution completed successfully. If \textcode{true}, then
  the execution was successful and the data stored in the
  \textidentifier{ResultField} is valid. If \textcode{false}, then the
  execution failed.
\item[\textcode{GetDataSet}] Returns a \textidentifier{DataSet} containing
  the results of the execution. The data set returned is a shallow copy of
  the input data with the generated field added.
\item[\textcode{GetField}] Returns the field information in a
  \vtkmcont{Field} object. Field objects are described in
  Section~\ref{sec:DataSets:Fields}.
\item[\textcode{FieldAs}] Given a \vtkmcont{ArrayHandle} object, allocates
  the array and copies the generated field data into it.
\end{description}

The following example provides a simple demonstration of using a field
filter. It specifically uses the point elevation filter, which is one of
the field filters.

\vtkmlisting[ex:PointElevation]{Using \textidentifier{PointElevation}, which is a field filter.}{PointElevation.cxx}

\subsection{Cell Average}

\index{cell~average|(}
\index{average|(}



\index{average|}
\index{cell~average|)}

\subsection{Point Elevation}

\index{point~elevation|(}
\index{elevation|(}

Example~\ref{ex:PointElevation}

\index{elevation|)}
\index{point~elevation|)}

\index{field~filter|)}
\index{filter!field|)}


\section{Data Set Filters}

\index{filter!data~set|(}
\index{data~set~filter|(}

\index{data~set~filter|)}
\index{filter!data~set|)}

\index{filter|)}
