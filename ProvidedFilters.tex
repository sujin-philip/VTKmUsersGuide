% -*- latex -*-

\chapter{Provided Filters}
\label{chap:ProvidedFilters}

\index{filter|(}

Filters are functional units that take data as input and write new data as
output. Filters operate on \vtkmcont{DataSet} objects, which are introduced
with the file I/O operations in Chapter~\ref{chap:IO} and are described in
more detail in Chapter~\ref{chap:DataSet}. For now we treat
\textidentifier{DataSet} mostly as an opaque object that can be passed
around readers, writers, filters, and rendering units.

\begin{didyouknow}
  The structure of filters in VTK-m is significantly simpler than their
  counterparts in VTK. VTK filters are arranged in a dataflow network
  (a.k.a. a visualization pipeline) and execution management is handled
  automatically. In contrast, VTK-m filters are simple imperative units,
  which are simply called with input data and return output data.
\end{didyouknow}

VTK-m comes with several filters ready for use, and in this chapter we will
give a brief overview of these filters. We group filters based on the type
of operation that they do and the shared interfaces that they have. Later
Part~\ref{part::Developing} describes the necessary steps in creating new
filters in VTK-m.


\section{Field Filters}

\index{filter!field|(}
\index{field~filter|(}

Every \vtkmcont{DataSet} object contains a list of \index{field}
\keyterm{fields}. A field describes some numerical value associated with
different parts of the data set in space. Fields often represent physical
properties such as temperature, pressure, or velocity. HEREHEREHERE.

\index{field~filter|)}
\index{filter!field|)}


\section{Data Set Filters}

\index{filter!data~set|(}
\index{data~set~filter|(}

\index{data~set~filter|)}
\index{filter!data~set|)}

\index{filter|)}
