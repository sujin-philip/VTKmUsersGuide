% -*- latex -*-

\chapter{Timers}
\label{chap:Timers}
\label{chap:Timer}

\index{timer|(}

It is often the case that you need to measure the time it takes for an
operation to happen. This could be for performing measurements for
algorithm study or it could be to dynamically adjust scheduling.

Performing timing in a multi-threaded environment can be tricky because
operations happen asynchronously. In the VTK-m control environment timing
is simplified because the control environment operates on a single
thread. However, operations invoked in the execution environment may run
asynchronously to operations in the control environment.

To ensure that accurate timings can be made, VTK-m provides a
\vtkmcont{Timer} class that is templated on the device adapter to provide
an accurate measurement of operations that happen on the device. If not
template parameter is provided, the default device adapter is used.

The timing starts when the \textidentifier{Timer} is constructed. The time
elapsed can be retrieved with a call to the \textcode{GetElapsedTime}
method. This method will block until all operations in the execution
environment complete so as to return an accurate time. The timer can be
restarted with a call to the \textcode{Reset} method.

\vtkmlisting{Using \protect\vtkmcont{Timer}.}{Timer.cxx}

\begin{commonerrors}
  Make sure the \textidentifier{Timer} being used is match to the device
  adapter used for the computation. This will ensure that the parallel
  computation is synchronized.
\end{commonerrors}

\begin{commonerrors}
  Some device require data to be copied between the host CPU and the
  device. In this case you might want to measure the time to copy data back
  to the host. This can be done by ``touching'' the data on the host by
  getting a control portal.
\end{commonerrors}

\index{timer|)}
